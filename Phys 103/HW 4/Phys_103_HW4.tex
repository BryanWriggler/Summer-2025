\documentclass{article}
\usepackage[margin = 2.54cm]{geometry} % set margin to traditional doc

%packages
\usepackage{graphicx} % Required for inserting images
\usepackage[most]{tcolorbox} %for creating environments
\usepackage{amsmath}
\usepackage{amssymb}
\usepackage{mathtools}
\usepackage{verbatim}
\usepackage[utf8]{inputenc}
\usepackage[dvipsnames]{xcolor} %for importing multiple colors
\usepackage{hyperref} %for creating links to different sections

\linespread{1.2} %controlling line spread

%define colors i like
\definecolor{myTeal}{RGB}{0,128,128}
\definecolor{myGreen}{RGB}{34,170,34}
\definecolor{mySapphire}{RGB}{15,82,186}
\definecolor{myEmerald}{RGB}{50.4, 130, 90}

%create math environments, can add [section] or [subsection] to add index counter based on sections/subsections
\newtheorem{define}{Definition}
\newtheorem{prop}{Proposition}
\newtheorem{thm}{Theorem}
\newtheorem{question}{Question}
\newtheorem{lemma}{Lemma}
\newtheorem{derive}{Derivation}

%setup colored box environment for each math env above
\tcolorboxenvironment{define}{
    enhanced, colframe=myTeal!50!teal, colback=myTeal!10,
    arc=5mm, lower separated=false, fonttitle=\bfseries, breakable
}
\tcolorboxenvironment{prop}{
    enhanced, colframe=myGreen!50!black, colback=myGreen!15,
    arc=5mm, lower separated=false, fonttitle=\bfseries, breakable
}
\tcolorboxenvironment{thm}{
    enhanced, colframe=mySapphire!50!mySapphire, colback=mySapphire!15,
    arc=5mm, lower separated=false, fonttitle=\bfseries, breakable
}
\tcolorboxenvironment{question}{
    enhanced, colframe=blue!50!black, colback=blue!10,
    arc=5mm, lower separated=false, fonttitle=\bfseries, breakable
}
\tcolorboxenvironment{lemma}{
    enhanced, colframe=myEmerald!50!myEmerald, colback=myEmerald!10,
    arc=5mm, lower separated=false, fonttitle=\bfseries, breakable
}
\tcolorboxenvironment{derive}{
    enhanced, colframe=myEmerald!50!myEmerald, colback=myEmerald!10,
    arc=5mm, lower separated=false, fonttitle=\bfseries, breakable
}

%setup hyperlink within pdf
\hypersetup{
    colorlinks=true,
    linkcolor=blue,
    filecolor=magenta,      
    urlcolor=cyan,
    pdftitle={Overleaf Example},
    pdfpagemode=FullScreen,
}

%common command (add to template)
%general
\newcommand{\FF}{\mathbb{F}}
\newcommand{\NN}{\mathbb{N}}
\newcommand{\ZZ}{\mathbb{Z}}
\newcommand{\QQ}{\mathbb{Q}}
\newcommand{\RR}{\mathbb{R}}
\newcommand{\CC}{\mathbb{C}}

\newcommand{\Id}{\textmd{Id}} %identity
\newcommand{\lcm}{\textmd{lcm}}
\DeclarePairedDelimiter{\abs}{\lvert}{\rvert}
\DeclarePairedDelimiter{\norm}{\lVert}{\rVert}
\DeclarePairedDelimiter{\paran}{(}{)}%paranthesis
\DeclarePairedDelimiter{\bracket}{\langle}{\rangle}

%algebra
\newcommand{\Gal}{\textmd{Gal}}
\newcommand{\Aut}{\textmd{Aut}}
\newcommand{\End}{\textmd{End}}
\newcommand{\Coker}{\textmd{Coker}}
\newcommand{\Hom}{\textmd{Hom}}
\newcommand{\Nil}{\textmd{Nil}}
\newcommand{\Char}{\textmd{char}}

%analysis
\newcommand{\Vol}{\textmd{Vol}}

%complex
\newcommand{\Real}{\textmd{Re}}
\newcommand{\Imag}{\textmd{Im}} %can also be used for Image
\newcommand{\Res}{\textmd{Res}}

%lie algebra
\newcommand{\gl}{\mathfrak{gl}}

%physics
\newcommand{\br}{\textbf{r}} %position
\newcommand{\bv}{\textbf{v}} %velocity
\newcommand{\ba}{\textbf{a}} %cceleration
\newcommand{\bF}{\textbf{F}} %force
\newcommand{\bP}{\textbf{P}} %momentum
\newcommand{\bL}{\textbf{L}} %angular momentum
\newcommand{\bN}{\textbf{N}} %torque
\newcommand{\bw}{\textbf{w}} %angular velocity
\newcommand{\bzero}{\textbf{0}}

%four-vectors for relativity
\newcommand{\bA}{\textbf{A}}
\newcommand{\bB}{\textbf{B}}
\newcommand{\bC}{\textbf{C}}
\newcommand{\bD}{\textbf{D}}


\title{Phys 103 HW4}
\author{Zih-Yu Hsieh}

\begin{document}
\maketitle

\section*{1 (does four-vector = Lorentz covariant vector?)}
\begin{question}\label{q1}
    An algebraic expression is said to be \emph{Lorentz covariant} if its form is the same in all inertial frame: the expression differs in two inertial frames $F$ and $F'$ only by putting prime marks on all the coordinate labels. For example, $C_\mu = \eta_{\mu\nu}B^\nu$ and $D=A^\mu \eta_{\mu\nu}B^{\nu}$ are both Lorentz covariant, as you could show using 
    $$\hat{\Lambda}^T\hat{\eta}\hat{\Lambda}=\hat{\eta}$$
    Lorentz scalars (which are unchanged), Lorentz vectors ($V^{\mu'} = \Lambda^{\mu'}_{\ \ \mu}V^\mu$), and Lorentzz tensors (as many factors of the transformation as appropriate) are thus all examples of Lorentz covariant quantities. (For all of these, the prefix may be \emph{four-} instead of Lorentz, or it may be dropped entirely if not trying to emphasize the four dimensions). If $K$ is a four-scalar and $\textbf{A},\textbf{B},\textbf{C},\textbf{D}$ are all four-vectors, identify each of the following expressions as a scalar, vector, tensor, or non-covariant (and prove/explain):
    \begin{itemize}
        \item $F=KA^\mu B^\mu$
        \item $F=K \eta_{\mu\nu}A^\mu B^\nu$
        \item $F^{\mu\nu} = KA^\mu B^\nu$
        \item $F^\lambda = C^\lambda A^\mu \eta_{\mu\nu}B^\nu$
        \item $F^\lambda = C^\mu A^\lambda \eta_{\mu\nu}B^\nu$
        \item $F=KA^\mu \eta_{\mu\nu}B^\lambda \eta_{\lambda\sigma}C^\nu D^\sigma$
    \end{itemize}
\end{question}

\textbf{Pf:}
\begin{itemize}
    \item For the expression $F=KA^\mu B^\mu$, it in general is non-covariant: If expressed in matrix multiplication, the term $A^\mu B^\mu$ is a dot product of the four-vectors $\textbf{A},\textbf{B}$, which in matrix form is as follow:
    \begin{align}
        F = K\textbf{A}^T\textbf{B}
    \end{align}
    Now, if change an inertial frame (the primed index) that has some relative speed with the original inertial frame (the unprimed index), then with $K$ being identical (since it's a Lorentz scalar / four-scalar), we get:
    \begin{align}
        F' = KA^{\mu'}B^{\mu'} = K(\hat{\Lambda}\textbf{A})^T (\hat{\Lambda}\textbf{B}) = K\textbf{A}^T(\hat{\Lambda}^T\hat{\Lambda})\textbf{B}
    \end{align}
    I general, $\hat{\Lambda}^T\hat{\Lambda}\neq \Id$, the identity matrix, so there exists four-vector $\textbf{B}$, such that $(\hat{\Lambda}^T\hat{\Lambda})\textbf{B}\neq \textbf{B}$ in terms of components (since $\hat{\Lambda}$ is symmetric, in index notation it is $\Lambda_{\lambda \sigma}\Lambda_{\sigma\mu}B^\mu \neq B^\lambda$). Similarly, there also exists four-vector $\textbf{A}$, such that $\textbf{A}^(\hat{\Lambda}^T\hat{\Lambda})\textbf{B}T\neq \textbf{A}^T\textbf{B}$ in terms of components. For instance, take $\textbf{A}=(1,0,0,0)$, $\textbf{B}=(1,1,0,0)$ (in unprimed frame $(ct,x,y,z)$), we get:
    \begin{align}
        &F = KA^\mu B^\mu = K\\
        &F' = K\textbf{A}^T(\hat{\Lambda}^T\hat{\Lambda})\textbf{B} = K\textbf{A}^T\hat{\Lambda}^T\begin{pmatrix}
            \gamma - \gamma\beta \\ \gamma - \gamma\beta \\ 0 \\ 0
        \end{pmatrix} = (\gamma-\gamma\beta)K\textbf{A}^T\hat{\Lambda}\textbf{B} = (\gamma-\gamma\beta)^2K\textbf{A}^T\textbf{B} = (\gamma-\gamma\beta)^2K
    \end{align}
    In general $(\gamma(1-\beta))^2\neq 1$ (since $(1-\beta)^2 \neq \paran{1-\beta^2}=\frac{1}{\gamma^2}$, the equality happens only at $\beta=0, \beta=1$, but $\beta=1$ implies $\frac{v}{c}=1$, which is not valid to say in speciall relativity), so for the above $F\neq F'$.

    \rule{15.6cm}{0,1mm}

    \item For $F=K \eta_{\mu\nu}A^\mu B^\nu = KA^\mu \eta_{\mu\nu}B^\nu$, it's a Lorentz Scalar: In matrix form it can be written as $F=K\textbf{A}^T\hat{\eta}\textbf{B}$. Which, under some arbitrary inertial frame (primed index) which is transformed from the original frame (unprimed index) using transformation $\hat{\Lambda}$, with $K$ being fixed, we get:
    \begin{align}
        F' = KA^{\mu'}\eta_{\mu'\nu'}B^{\nu'} = K(\hat{\Lambda}\textbf{A})^T\hat{\eta}(\hat{\Lambda}\textbf{B}) = K\textbf{A}^T(\hat{\Lambda}^T\hat{\eta}\hat{\Lambda})\textbf{B} = K\textbf{A}^T\hat{\eta}\textbf{B} = F
    \end{align}
    So, under arbitrary Lorentz transformation, such term is fixed as a scalar, which is a Lorentz scalar.

    \rule{15.6cm}{0,1mm}

    \item For $F^{\mu\nu} = KA^\mu B^\nu$, it in general is non-covariant: In matrix form, let $\hat{F}$ represents the matrix $(F^{\mu\nu})$ (with row index $\mu$ column index $\nu$), we get $\hat{F} = K\textbf{A}\textbf{B}^T$. Which, under a different inertial frame (primed index) with some relative speed with the original frame (unprimed index), as $K$ is identical, we get:
    \begin{align}
        F^{\mu'\nu'} = KA^{\mu'}B^{\nu'},\quad \hat{F'}:= (F_{\mu'\nu'}),\ \hat{F'} = K(\hat{\Lambda}\textbf{A})(\hat{\Lambda}\textbf{B})^T  = K\hat{\Lambda}\textbf{A}\textbf{B}^T\hat{\Lambda}^T
    \end{align}
    Which, can choose $\textbf{A},\textbf{B}$ such that $\hat{\Lambda}\textbf{A}\textbf{B}^T\hat{\Lambda}\neq \textbf{A}\textbf{B}^T$. For instance, let $\textbf{A}=(1,0,0,0),\textbf{B}=(1,1,0,0)$ (in unprimed frame $(ct,x,y,z)$), we get:
    \begin{align}
        \textbf{A}\textbf{B}^T = \begin{pmatrix}
            1&1&0&0\\0&0&0&0\\0&0&0&0\\0&0&0&0
        \end{pmatrix}
    \end{align}
    \begin{align}
        \hat{\Lambda}\textbf{A} = \begin{pmatrix}
            \gamma\\-\gamma\beta\\0\\0
        \end{pmatrix},\quad \hat{\Lambda}\textbf{B} = \begin{pmatrix}
            \gamma-\gamma\beta\\\gamma-\gamma\beta\\0\\0
        \end{pmatrix} = \gamma(1-\beta)\textbf{B}
    \end{align}
    \begin{align}
        \hat{\Lambda}\textbf{A}\textbf{B}^T\hat{\Lambda}^T = \begin{pmatrix}
            \gamma\\-\gamma\beta\\0\\0
        \end{pmatrix}\begin{pmatrix}
            \gamma(1-\beta) & \gamma(1-\beta)&0&0
        \end{pmatrix} = \begin{pmatrix}
            \gamma^2(1-\beta) & \gamma^2(1-\beta)&0&0\\
            -\gamma^2\beta(1-\beta) & -\gamma^2\beta(1-\beta)&0&0\\
            0&0&0&0\\0&0&0&0
        \end{pmatrix}
    \end{align}
    Since in general $\gamma^2\beta(1-\beta)\neq 0$ (unless $\beta=0,1$, which $\beta=1$ is not possible), then the two matrices are not the same. Hence, in general $F^{\mu\nu}\neq F^{\mu'\nu'}$, showing that it is non-covariant.

    \rule{15.6cm}{0,1mm}

    \item For $F^\lambda = C^\lambda A^\mu \eta_{\mu\nu}B^\nu$, it is non-covariant: under matrix representation, let $\textbf{F}$ be the four-vector with components $F^{\mu}$, then we get $\textbf{F} = (A^{\mu}\eta_{\mu\nu}B^{\nu})\textbf{C}$. From the problem, we know $D=A^\mu\eta_{\mu\nu}B^{\nu}$ is a Lorentz scalar, which is invariant under Lorentz transformation (also proven in the second part); which, under a different inertial frame (primed index) with some relative speed to the original frame (unprimed index), we get:
    \begin{align}
        F^{\lambda'} = C^{\lambda'}(A^{\mu'}\eta_{\mu'\nu'}B^{\nu'}) = C^{\lambda'}(A^\mu\eta_{\mu\nu}B^{\nu}) \implies \hat{\Lambda}\textbf{F} = (A^\mu\eta_{\mu\nu}B^\nu) \hat{\Lambda}\textbf{C}
    \end{align}
    Which, there exists $\textbf{C}$, where in terms of components, $\hat{\Lambda}\textbf{C}\neq \textbf{C}$. For instance, with $\textbf{C}=(1,0,0,0)$ (in unprimed frame $(ct,x,y,z)$), we get:
    \begin{align}
        \hat{\Lambda}\textbf{C} = \begin{pmatrix}
            \gamma\\-\gamma\beta\\0\\0
        \end{pmatrix} \neq \textbf{C}
    \end{align}
    Which, with $\hat{\Lambda}\textbf{F} = =(A^\mu\eta_{\mu\nu}B^\nu)\hat{\Lambda}\textbf{C}$ and $\textbf{F}=(A^\mu\eta_{\mu\nu}B^\nu)\textbf{C}$, in general $\hat{\Lambda}\textbf{F}\neq \textbf{F}$, showing it's non-covariant.

    \rule{15.6cm}{0,1mm}

    \item For $F^\lambda = C^\mu A^\lambda \eta_{\mu\nu}B^\nu = A^\lambda(C^\mu\eta_{\mu\nu}B^\nu)$, notice that swqp $\textbf{A}$ and $\textbf{C}$, it is identical to the fourth one (right above), so it is also non-covariant.
    
    \rule{15.6cm}{0,1mm}

    \item For $F=KA^\mu \eta_{\mu\nu}B^\lambda \eta_{\lambda\sigma}C^\nu D^\sigma = K(A^\mu\eta_{\mu\nu}C^\nu)(B^\lambda\mu_{\lambda\sigma}D^\sigma)$, since in the problem with any four vectors $\textbf{U},\textbf{V}$, the term $U^\mu\eta_{\mu\nu}V^\nu$ is a Lorentz scalar (also proven in the second part), then $F$ is in fact a product of three Lorentz scalars. In any frame, these three scalars are fixed, so $F$ is also fixed, showing that it's also a Lorentz scalar.
\end{itemize}

\break

\section*{2 (Not done, Insert image)}
\begin{question}\label{q2}
    Al and Bert are identical twins. When Bert is 24 years old, he travels to a distant planet at speed $\frac{12}{13}c$, turns around, and heads back at the same peed, arriving home at age $44$. Al stays at home. How old is Al when Bert returns? How far away was the planet in Al's frame? Why can't Bert reasonably claim that from his point of view it was Al who was moving, so that Al's clocks should be dilated, making Al younger than Bert when they reunite? (Make sure to give a reasonably thorough ansewr; rather than just a quick statement, you should consider what's going on in some different frames and why certain calculations or interpretations fail).
\end{question}

\textbf{Pf:}

For this problem, we can make a few premises: First, assume Bert is traveling in $x$-direction in Al's frame (unprimed), while during the forward trip Bert's frame is denoted using $x'$ (primed), while the backward trip Bert's frame is denoted using $x''$ (double primed). And, for simplicity, we'll use year for time scale.

\textbf{Insert spacetime diagram}

First, since Bert travels to a planet and comes back, in Al's frame Bert travels through two identical distance; and, since for both trips, Bert's relative speed to Al is $\frac{12}{13}c$ (in different direction), both trips (except the part of changing direction) are identical. Hence can assume for both trips in Bert's frame, he travels the same distance, while both trips take the same time in Bert's frame. Then, since for the two trips, in Bert's frame he changes from 24 to 44 years old, so in Bert's frame he passes through total of 20 years, with the two trips being similar (only different in direction), each trip takes $\Delta t' = 10$ years in Bert's frame. 

Using the first trip as reference, Bert observes the planet heading towards him with speed $\frac{12}{13}c$, hence in Bert's frame, the planet travels through distance $\Delta x' = \frac{12}{13}c \cdot 10$, where it's with unit m/s times year.

Here, if apply Length Contraction, with speed $\frac{12}{13}c$, we get $\beta = \frac{12}{13}c/c = \frac{12}{13}$, and $\gamma = \frac{1}{\sqrt{1-\beta^2}} = \frac{13}{5}$. Since it is the distance that is traveling in Bert's frame, if the original distance (between Al and the planet) is $\Delta x$, by length contraction $\Delta x' = \frac{\Delta x}{\gamma}$, or $\Delta x=\gamma\Delta x' = \frac{13}{5}\frac{120}{13}c = 24c$ m/s times year.

Or, in Bert's frame, the event of him reaching the planet has coordinates $(ct', x') = (c\Delta t', \Delta x') = (10c, 0)$ (both are with unit of m/s times year; since in Bert's frame he never moves, so $x'=0$). Hence, if this event of Bert reaching the planet has coordinate $(ct, x)$ in Al's frame, using Lorentz boost, we get:
\begin{align}
    \begin{pmatrix}
        10c\\
        0
    \end{pmatrix} = \begin{pmatrix}
        ct'\\x'
    \end{pmatrix} = \begin{pmatrix}
        \gamma & -\gamma\beta\\
        -\gamma\beta & \gamma
    \end{pmatrix}\begin{pmatrix}
        ct\\x
    \end{pmatrix} = \begin{pmatrix}
        \frac{13}{5}\paran*{ct - \frac{12}{13}x}\\
        \frac{13}{5}\paran*{x - \frac{12}{13}ct}
    \end{pmatrix} = \begin{pmatrix}
        \frac{13}{5}ct - \frac{12}{5}x\\
        \frac{13}{5}x - \frac{12}{5}ct
    \end{pmatrix}
\end{align}
(Note: above both have units of m/s times year).

Which, solving the system of equations, we get (both in units of m/s times year):
\begin{align}
    x= 24c,\quad ct = \frac{313}{13}c
\end{align}
Which, this is the coordinate of Bert reaching the planet in Al's frame. The answer of $x$ coincides with length contraction, showing it's valid to apply here. Transfer the distance to light year, in Al's frame the planet has coordinate $x=24$ light years.

Hence, with Bert's speed being $\frac{12}{13}c$ m/s, in Al's frame, it takes Bert $\Delta t = \frac{24c}{\frac{12}{13}c} = 26$ years to reach the planet. And, since for Bert's trip back it's a similar system, it takes a total of $\Delta t=52$ year for Bert to travel back and forth in Al's frame. At this point, in Al's frame himself is $24 + 52 = 76$ years old.

\hfil

The reason why Bert can't claim that Al's clock is the one dilated (which supposedly should cause Al to be younger than Bert when reunite), is because Bert is not in an inertial frame: Even though we assumed the change of direction to be happen in an instant, but realistically during that moment Bert is in fact accelerating with respect to Al and the Planet, which makes him no longer in an inertial frame, and cause the calculation to be more complicated.

\break

\section*{3 (Insert image)}
\begin{question}\label{q3}
    Cookie dough lies on a conveyer belt that moves at speed $v$. A circular stamp of radius $r$ stamps out cookies as the dough rushes by beneath it. When you buy these cookies in a store, what shape are they? If you think they're not circular, be sure to say (and explain) whether they're squashed or stretched relative to the samp. Explain what's going on from both cookie's and the stamp's perspective.
\end{question}

\textbf{Pf:}

For this question, assume in cookie dough's frame the cookie never deforms (i.e. in either frame of the stamp or the cookies, the cookies never change its size). 

Also, it's important to verify wich frame it is when the stamp touches the cookies: Here, we'll assume that the stamp lands on the cookie "simultaneously" in the stamp's frame (i.e. the whole area of the stamp lands on the cookie at the same time $t$ in the stamp's frame).

We'll assume that the cookie dough is traveling forward, meaning $x$ direction of the stamp's frame, and $x'$ direction of the cookie's frame.

\subsection*{In the Stamp's Frame:}
Since we assume that the stamp's area touches the dough simultaneously in the stamp's frame, then the important part is the dough's length (in traveling direction) in the stamp's frame. 

If assume both frames are inertial (or the relative speed never changes), then with $\Delta t=0$ in the stamp's frame, length contraction can be applied: Since the dough is moving relative to the stamp, then the length of the dough has contracted in the stamp's frame (while the width in the orthogonal direction doesn't). So, the stamp would cover a larger portion of the length of the dough, while the width wasn't affected. Then, when the cookies are bought in a store, the length of the stamp would be longer, showing a shape of an elllipse (or, the circular stamp is stretched in the moving direction on the dough).

\textbf{Insert explanation image}

\subsection*{In the Cookie Dough's Frame:}
Intuitively, if we think about the cookie dough's frame, because the stamp is the one moving, then applying length contraction, the stamp should have a length shorter than usual, so if "simultaneously" stamp the stamp on the cookie dough, it should be an ellipse with the length shrinked (like below):

\textbf{insert false image}

But, the problem is, if we assume stamping on the dough "simultaneously" on the dough in the stamp's frame, this set of events (where the points on the stamp touches the points on the dough) appears to not be "simultaneous" in the cookie dough's frame, which can be seen from the following spacetime diagram:

\textbf{Insert Spacetime diagram}

Because the stamping process doesn't appear to be "simultaneous" in the cookie dough's frame, then we can't directly apply length contraction here, which causes no contradiction from the previous part.

\hfil

Overall, we can say that the stamp appears to be an ellipes, which is stretched in the cookie dough's traveling direction (if we assume the stamping process is "simultaneous" in the stamp's frame).

\break

\section*{4 (Table Not done, Insert image)}
\begin{question}\label{q4}
    A stick of proper length $L$ moves at speed $v$ in the direction of its length. It passes over an infinitesimally thin sheet that has a hole of diameter $L$ cut in it. As the stick passes over the hole, the sheet is raised so that the stick passes through the hole and ends up underneath the sheet. Well, maybe\dots

    In the lab frame, the stick's length is contracted to$L/\gamma$, so it appears to easily make it through the hole. But in the stick frame, the \emph{hole} is contracted to $L/\gamma$, so it appears that the stick does \emph{not} make it through the hole (or rather, the hole doesn't make it around the stick, since the hole is what is moving in the stick frame). So the question is: does the stick end up on the other side of the sheet or not?

    Once you have that question figured out, also consider this variant: suppose it's not a sheet, but a table (still with a hole of proper length $L$). The stick slides along the table at speed $v$; does it make it across the gap?
\end{question}

\textbf{Pf:}

\subsection*{Thin Sheet Problem:}
Here, it is more important to understand to understand the simultaneity when the stick passes through the hole (like in Question \ref{q3}, the simultaneity of the stamping process). Based on the description, since when the stick passes over the hole, the sheet is raised for the stick to pass through the hole, can assume every part of the stick simultaneously passes through the hole in the lab frame.

So, using the lab frame, with the stick's length contracted to $L/\gamma<L$, the stick would have no problem passing through the hole.

\hfil

If we turn to the stick's frame, intuitively using length contraction, we'd get that the hole contracts to length $L/\gamma<L$, so it's not possible for the stick to pass through the hole "simultaneously" in the stick's frame; however, because the simultaneity in the lab frame records a different set of events (or different set of the parts of the sticks) in spacetime, being simultaneous in the lab frame is different from being simultaenous in the stick's frame. So, the stick is not passing through the hole "simultaneously" in the stick's frame. This can also be seen in the following spacetime diagram:

\textbf{Insert Spacetime Diagram}

\subsection*{Table Problem:}


\break

\section*{5 (Not done, need other methods, Insert image)}
\begin{question}\label{q5}
    Two spaceships float in space and are at rest relative to each other a distance $\ell$ apart. They are connected by a string of unstretched length $\ell$. The string can stretch somewhat, but it can't withstand an arbitrary amount of stretching. In the inertial frame which is initially the rest frame of the spaceships, at a given instant, the spaceships simultaneously begin accelerating in the same direction (along the line between them) with uniform acceleration. (In other words, assume they have identical engines and put them on the same setting). Consider the following two arguments:
    \begin{itemize}
        \item Take the entire contraption of the two ships and the string, considered to be one object. The proper length of this object is $\ell$. Therefore, when it's moving at some speed $v$, its length in the initial rest frame is contracted to $\ell/\gamma$. The string is thus always at its unstretched length, no matter how fast the ships are going, so it never breaks.
        \item Since they start at the same speed and have the same acceleration, the two ships are always moving at the same speed. Therefore, the distance between them (in the initial frame) is always $\ell$. When the string is moving at speed $v$, its length when unstretched in $\ell/\gamma$; it is therefore stretched by a factor $\gamma$ as long as it connects the two ships. After enough time, $\gamma$ will be large enough that the string breaks.
    \end{itemize}
    Which is correct? Or are both wrong? Make sure to support your answer with a thorough argument; in particular, you might consider what's going on in \emph{other} frames.
\end{question}

\textbf{Pf:}

First, we'll consider the worldline of the two spaceships: In the inertial frame (denote as $F$), we can assume at $t=0$, the first spaceship has $x$ coordinate $x_1(0)=0$, then the second spaceship has $x$ coordinate $x_2(0)=\ell + x_1(0) = \ell$. Then, because both spaceships are at rest when $t=0$, and both has uniform acceleration $a$, then we get that $v_1(t)=v_2(t) = at$ in frame $F$ (which, $v(t)<c$, showing that $t < \frac{c}{a}$). Then, we get that $x_1(t) = \frac{1}{2}at^2$ and $x_2(t) = \frac{1}{2}at^2 + \ell$ based on the initial conditions. Hence, with the $ct$ and $x$ coordinates, the set $\{(ct,\frac{1}{2}at^2)\ |\ t\geq 0\}$ is the worldline of the first spaceship, while $\{(ct,\frac{1}{2}at^2+\ell)\ |\ t\geq 0\}$ is the worldline of the second spaceship. This also shows that the distance between the two spaceships is $\ell$ at any time $t\geq 0$ (in frame $F$).

\hfil

Then,to consider the change in rope's length, we need to consider the simultaneity of a frame of one of the spaceships (we'll use the first spaceship here): In frame $F$, for any time $t_0\geq 0$, the coordinates of spaceship $1$ is $(ct_0, \frac{1}{2}at_0^2)$ and with speed $v=at_0$ in $x$ direction. Which, using Lorentz Boost, we get that the difference of events from spaceship 1 in each coordinate of spaceship's frame is as follow:
\begin{align}
    \begin{pmatrix}
        c\Delta t'\\\Delta x'
    \end{pmatrix} = \begin{pmatrix}
        \gamma & -\gamma\beta\\ -\gamma\beta&\gamma
    \end{pmatrix}\begin{pmatrix}
        c\Delta t\\\Delta x
    \end{pmatrix} = \begin{pmatrix}
        \gamma c\Delta t-\gamma\beta \Delta x\\ -\gamma\beta c\Delta t + \gamma\Delta x
    \end{pmatrix}
\end{align}
If we want to know the length of the rope at each instant in the spaceship 1's frame, we need $\Delta t'=0$ (while $\Delta x'\in\RR$ can be arbitrary). Which, with $0=c\Delta t' = \gamma c\Delta t-\gamma\beta \Delta x$, we get $c\Delta t = \beta\Delta x$; and, with the coordinates of spaceship 1 being $(ct_0,\frac{1}{2}at_0^2)$ in frame $F$, then we get the following line as the set of events happening "simultaneously" in spaceship 1's frame:
\begin{align}
    c(t - t_0) = \beta\paran*{x-\frac{1}{2}at_0^2} = \frac{v}{c}\paran*{x-\frac{1}{2}at_0^2} = \frac{at_0}{c}\paran*{x-\frac{1}{2}at_0^2}
\end{align}
Which, suppose at some unknown time $t=t_0+\Delta t$ (in frame $F$), spaceship 2 happens to be in one of these events, then plug in time $t$ and position $\frac{1}{2}at^2+\ell$, we get:
\begin{align}
    &c((t_0+\Delta t)-t_0) = \frac{at_0}{c}\paran*{\frac{1}{2}a(t_0+\Delta t)^2+\ell - \frac{1}{2}at_0^2}\\
    &\implies c\Delta t = \frac{at_0\ell}{c} + \frac{a^2t_0}{2c}\Delta t(2t_0+\Delta t)\\
    &\implies c\Delta t = \frac{a^2t_0}{2c}(\Delta t)^2 + \frac{a^2t_0^2}{c}\Delta t+\frac{at_0\ell}{c}\\
    &\implies \frac{a}{2}\beta(\Delta t)^2 + \paran*{\frac{v^2}{c}-c}\Delta t + \beta\ell = 0\\
    &\implies \frac{a\beta}{2}(\Delta t)^2 + c\paran*{\beta^2-1}\Delta t + \beta \ell = 0\\
    &\implies (\Delta t)^2 - \frac{2c}{a\beta\gamma^2}\Delta t + \frac{2\ell}{a}=0\\
    &\implies \Delta t = \frac{c}{\alpha\beta\gamma^2} \pm \sqrt{\frac{c^2}{a^2\beta^2\gamma^4}-\frac{\ell}{a}}
\end{align}
Which, take the smaller value (need to explain what happens with the two values).

\textbf{Insert spacetime diagram to help explain}

Because this event happens "simultaneously" with spaceship 1 in the spaceship's frame, then it satisfies $c\Delta t = \beta \Delta x$ (or $\frac{c}{\beta}\Delta t=\Delta x$). Hence, when calculating the Lorentz Invariant scalar, we get:
\begin{align}
    (\Delta s)^2 = -c^2(\Delta t)^2+(\Delta x)^2 = \paran*{\frac{c^2}{\beta^2}-c^2}\paran*{\frac{2c^2}{a^2\beta^2\gamma^4}-\frac{\ell}{a}-\frac{2c}{a\beta\gamma^2}\sqrt{\frac{c^2}{a^2\beta^2\gamma^4}-\frac{\ell}{a}}}
\end{align}

\break

\section*{6}
\begin{question}\label{q6}

    \hfil

    \begin{itemize}
        \item[(a)] Prove that the time order of two events (i.e. which one occurs at an earlier time) is the same in all inertial frames iff they can be connected by a signal traveling at or below speed $c$.
        \item[(b)] Suppose that in an inertial fram $F$ a particular signal from $A$ to $B$ can travel at velocity $v=2c$. Show that there exists an inertial fram $F'$ (whose speed relative to $F$ is less than the speed of light) in which the signal arrives at $B$ before it is sent from $A$.
    \end{itemize}
\end{question}

\textbf{Pf:}

\subsection*{(a)}
\begin{itemize}
    \item[$\implies$:] First, suppose the time order of two events $\bA, \bB$ is the same in all inertial frames (WLOG, can assume that in all inertial frame, $\bA$ happens before or simultaneous with $\bB$). 
    
    Now, fix an inertial frame $F$ (using unprimed coordinates) with coordinates such that $\bA,\bB$ only differ in $x$-component (suppose $\bA$ has $x$-coordinate $x$, and $\bB$ has $x$-coordinate $x+\Delta x$ for some $x,\Delta x\in\RR$). 
    Consider a Lorentz Boost in the $x$ direction with arbitrary speed $0< v<c$, then in the new inertial frame $F'$ (using primed coordinates), we get the following relation with time difference $\Delta t'$:
    \begin{align}
         \gamma c\Delta t-\gamma\beta \Delta x =c\Delta t'\geq 0\implies c\Delta t \geq \beta \Delta x
    \end{align}
    (Note: $\Delta t' = t_B' - t_A' \geq 0$ based on our assumed time order of $\bA,\bB$).

    Similarly, consider a Lorentz Boost in the $-x$ direction with the same speed $v$ from the original frame $F$, then in the new inertial frame $F''$ (using double primed coordinates), we get the following relation with time difference $\Delta t''$:
    \begin{align}
        \gamma c\Delta t + \gamma\beta \Delta x = c\Delta t'' \geq 0\implies c\Delta t \geq -\beta \Delta x
    \end{align}
    With $|\Delta x| = \Delta x$ or $|\Delta x| = -\Delta x$ (at least one equation is true), we have $c\Delta t\geq \beta |\Delta x|$.

    Here, there are two situations:
    \begin{itemize}
        \item If $\Delta t=0$, then with $v>0$, $\beta =\frac{v}{c}>0$, hence $0=c\Delta t \geq \beta |\Delta x|\geq 0$ enforces $\beta |\Delta x|=0$, or $|\Delta x|=0$. Therefore, $\Delta x = 0$.
        
        Since we assume $\bA,\bB$ has spatial coordinates in $F$ differ by only $x$ component, then with $\Delta x=0$ and $\Delta t=0$, this implies $\bA=\bB$ (they're the same event in spacetime). Then, they can trivially be connected by a signal traveling at or below speed $c$.

        \item Else, if $\Delta t>0$, then with $\beta >0$ (proven above), the inequality becomes:
        \begin{align}
            c> v = \frac{c}{\beta}\geq \frac{|\Delta x|}{\Delta t} \geq 0
        \end{align}
        Then, choose a signal with speed $u = \frac{|\Delta x|}{\Delta t}$ that propogates from the spatial coordinate of $\bA$ to spatial coordinate of $\bB$ (either in $x$ or $-x$ direction, depending on the sign of $\Delta x$), if the signal starts propogating at time $t$ (when event $\bA$ happens in frame $F$), then in frame $F$ after time $\Delta t$, it reaches the spatial coordinates of $\bB$, and the time $t+\Delta t$ is when event $\bB$ happens in frame $F$. So, this signal can propogate from $A$ to $B$, while its speed $0\leq u = \frac{|\Delta x|}{\Delta t}<c$.
    \end{itemize}
    In conclusion, the above two cases show that "time order of two events is the same in all inertial frames" $\implies$ "they can be connected by a signal traveling at or below speed $c$".

    \rule{15.6cm}{0.1mm}

    \item[$\impliedby$:] Now, suppose the events $\bA,\bB$ can be connected by a signal traveling at or below speed $c$. Fix an inertial frame $F$ (unprimed), and WLOG, assume $\Delta t = t_B - t_A > 0$ (i.e. $\bA$ happens before $\bB$ in this inertial frame). 
    
    (Note: can assume $\Delta t>0$, since if $\Delta t = 0$, then in frame $F$ there's time $\Delta t=0$ for the signal to propogate, which the signal reaches nowhere but the spatial coordinates of $\bA$, then if the signal reaches $\bB$'s spatial coordinates within no time, $\bA$ and $\bB$ must have the same spatial coordinates, or $\bA=\bB$ since they also have the same time coordinate).

    Which, there exists a signal with some speed $0\leq u\leq c$ in frame $F$, that propagates from spatial position of $\bA$ (starting at $t_A$ in frame $F$) to the spatial position of $\bB$ (ending at $t_B$ in frame $F$).
    
    \hfil

    Now, consider an inertial frame $F'$ with speed $0<v<c$ relative to $F$. Choose the coordinate of $F$ and $F'$, such that the $x$ direction aligns with the relative velocity of frame $F'$ from frame $F$. In this coordinates of frame $F$, $\bA=(ct_A, x,y,z)$ for some $x,y,z\in \RR$, while $\bB = (c(t_A+\Delta t),x+\Delta x,y+\Delta y,z+\Delta z)$ for some $\Delta x,\Delta y,\Delta z\in\RR$. Then, with the signal propogates from $\bA$ to $\bB$ with displacement $(\Delta x,\Delta y,\Delta z)$ and time $\Delta t>0$ in frame $F$, we get the following:
    \begin{align}
        &\Delta r=\sqrt{(\Delta x)^2+(\Delta y)^2+(\Delta z)^2},\quad u = \frac{\Delta r}{\Delta t} \leq c\\
        &\implies (\Delta x)^2\leq (\Delta x)^2+(\Delta y)^2 + (\Delta z)^2=(\Delta r)^2\leq c^2(\Delta t)^2\\
        &\implies |\Delta x|\leq c|\Delta t| = c\Delta t
    \end{align}
    (Note: $\Delta r$ is the distance the signal travels).

    Which, under Lorentz Trnasformation from $F$ to $F'$, we get:
    \begin{align}
        \begin{pmatrix}
            c\Delta t'\\ \Delta x'\\ \Delta y'\\ \Delta z'
        \end{pmatrix} = \begin{pmatrix}
            \gamma & -\gamma\beta & 0&0\\
            -\gamma\beta & \gamma &0&0\\
            0&0&1&0\\0&0&0&1
        \end{pmatrix}\begin{pmatrix}
            c\Delta t\\\Delta x\\\Delta y\\\Delta z
        \end{pmatrix} = \begin{pmatrix}
            \gamma c\Delta t-\gamma\beta \Delta x\\
            -\gamma\beta c\Delta t + \gamma \Delta x\\
            \Delta y\\
            \Delta z
        \end{pmatrix}
    \end{align}
    Which, with the inequality $-|\Delta x|\leq -\Delta x$, $0<\beta = \frac{v}{c}<1$, and $|\Delta x|\leq c\Delta t$, we get the following inequality:
    \begin{align}
        &c\Delta t - \beta \Delta x \geq c\Delta t - \beta |\Delta x| > c\Delta t - |\Delta x| \geq 0\\
        &\implies \Delta t' = \frac{\gamma}{c}(c\Delta t-\gamma\beta \Delta x) > 0
    \end{align}
    Hence, we also have $t'_B - t'_A = \Delta t' >0$, showing that in frame $F'$, $\bB$ still happens after $\bA$, they still have the same time order as in frame $F$. Then, with $F'$ being an inertial frame with arbitrary nontrivial relative velocity to frame $F$ (the original frame of reference), we can claim that in all valid inertial frames, the time order of $\bA,\bB$ is preserved.

    In conclusion, "Two events can be connected by a signal traveling at or below speed $c$" $\implies$ "The time order of the two events is the same in all inertial frames".
\end{itemize}

\subsection*{(b)}
Suppose in an inertial frame $F$, a signal from $\bA$ to $\bB$ can travel at velocity $v=2c$. WLOG, can set the coordinates so that the velocity is in $x$ direction for simplicity. Also, based on the problem description, can assume that $\Delta t:= t_B-t_A >0$ (i.e. in this frame, $\bA$ happens before $\bB$). 

Then, let $\bA = (ct_A, x,y,z)$ in coordinates of frame $F$, $\bB = (c(t_A+\Delta t), x+2c\Delta t, y, z)$ in coordinates of frame $F$ (since the time difference of two events is $\Delta t$, and the signal travels at speed $2c$ in $x$ direction, which only has a displacement in $x$, with distance $2c\Delta t$).

\hfil

Now, choose a Lorentz boost. in $x$ direction with speed $v = \frac{4}{5}c$ (which $\beta = \frac{v}{c}=\frac{4}{5}$, $\sqrt{1-\beta^2} = \frac{3}{5}$, and $\gamma = \frac{1}{\sqrt{1-\beta^2}}=\frac{5}{3}$). Then, after this transformation, with $\Delta x = (x+2c\Delta t)-x = 2c\Delta t$, we get the following for $\Delta t'$:
\begin{align}
    &c\Delta t' = \gamma c\Delta t - \gamma\beta \Delta x  = \frac{5}{3}\paran*{c\Delta t - \frac{4}{5}\cdot 2c\Delta t} = \frac{5}{3}c\Delta t\paran*{1-\frac{8}{5}} = -c\Delta t\\
    &\implies \Delta t' = -\Delta t<0
\end{align}
Hence, in the frame $F'$, $t'_B-t'_A = \Delta t' <0$, or $t'_B < t'_A$, showing that event $\bB$ happens before event $\bA$. Hence, the signal must arrive at $\bB$ before it is sent from $\bA$, this proves the existence of such frame (where $\bB$ receives the signal before the signal has sent from $\bA$).

\break

\section*{7}
\begin{question}\label{q7}
    A wave equation for a wave traveling at the speed of light is 
    $$\frac{\partial^2\phi}{\partial x^2}+\frac{\partial^2\phi}{\partial y^2}+\frac{\partial^2\phi}{\partial z^2}-\frac{1}{c^2}\frac{\partial^2\phi}{\partial t^2}=0$$
    Where $\phi$ is a Lorentz scalar. Show that the wave equations takes the same form after applying any of the \emph{Poincare transformations}, which consist of the Lorentz transformations and translations in each of the four directions. To avoid some algebra, you may find it useful to define 
    $$\partial_\mu \equiv \frac{\partial}{\partial r^\mu}$$
    How does $\partial_\mu$ transform under a Lorentz transformation?
\end{question}

\textbf{Pf:}

Notice that if this equation is true under both Lorentz transformations and translations in spactime, then it's also true for arbitrary compositions of the two, hence the statement would be true for all Poincare transformations.

\subsection*{1) Translations:}
Suppose the coordinates is transformed as:
\begin{align}
    (ct', x', y', z') = (c(t+t_0), x+x_0, y+y_0, z+z_0)
\end{align}
for some $x_0,y_0,z_0\in\RR$, then since $\frac{\partial r^{\mu'}}{\partial r^\mu} = 1$ for all index $\mu$, we get that $\frac{\partial \phi}{\partial r^\mu} = \frac{\partial \phi}{\partial r^{\mu'}}\frac{\partial r^{\mu'}}{\partial r^\mu} = \frac{\partial \phi}{\partial r^{\mu'}}$. Hence, the wave equation is of the same form in the frame after translation.

\subsection*{2) Lorentz Boosts:}
For a more simple case, we'll consider the Lorentz Boost with speed $0<v<c$ in $x$ direction. Recall that its coordinates transformation is given as:
\begin{align}
    \begin{pmatrix}
        ct'\\x'\\y'\\z'
    \end{pmatrix} = \begin{pmatrix}
        \gamma & -\gamma\beta & 0&0\\
        -\gamma\beta & \gamma &0&0\\
        0&0&1&0\\0&0&0&1
    \end{pmatrix}\begin{pmatrix}
        ct\\x\\y\\z
    \end{pmatrix} = \begin{pmatrix}
        \gamma (ct)-\gamma\beta x\\ -\gamma\beta (ct) + \gamma x\\y\\z
    \end{pmatrix}
\end{align}
Hence, define $r^t := ct$, which for any differentiable real-valued function $\phi$ with input $(t,x,y,z)$, $\frac{\partial \phi}{\partial t}=\frac{\partial\phi}{\partial r^t} \frac{\partial r^t}{\partial t} = c\frac{\partial \phi}{\partial r^t}$ (or $\frac{\partial \phi}{\partial r^t}=\frac{\partial\phi}{\partial (ct)}=\frac{1}{c}\frac{\partial\phi}{\partial t}$), the following equality true as operators:
\begin{align}
    \frac{\partial\phi}{\partial r^\mu} = \frac{\partial\phi}{\partial r^{\mu'}}\frac{\partial r^{\mu'}}{\partial r^\mu},\quad \begin{cases}
        \frac{\partial\phi}{\partial (ct)} = \gamma\frac{\partial\phi}{\partial (ct')}-\gamma\beta\frac{\partial\phi}{\partial x'}\\
        \frac{\partial \phi}{\partial x} = -\gamma\beta\frac{\partial \phi}{\partial (ct')} + \gamma\frac{\partial\phi}{\partial x'}\\
        \frac{\partial\phi}{\partial y} = \frac{\partial\phi}{\partial y'}\\
        \frac{\partial\phi}{\partial z} = \frac{\partial\phi}{\partial z'}
    \end{cases}
\end{align}
For twice-differentiable functions (like waves for instance), we get:
\begin{align}
    \begin{cases}
        \frac{1}{c^2}\frac{\partial^2\phi}{\partial t^2}=\frac{\partial^2 \phi}{\partial (ct)^2} = \gamma^2\frac{\partial^2\phi}{\partial (ct')^2} + \gamma^2\beta^2\frac{\partial^2\phi}{\partial (x')^2} - \gamma^2\beta\frac{\partial^2\phi}{\partial(ct')\partial (x')}\\
        \frac{\partial^2\phi}{\partial x^2} = \gamma^2\beta^2\frac{\partial^2\phi}{\partial (ct')^2} + \gamma^2\frac{\partial^2\phi}{\partial (x')^2}-\gamma^2\beta\frac{\partial^2\phi}{\partial (ct')\partial (x')}\\
        \frac{\partial^2\phi}{\partial y^2}=\frac{\partial^2\phi}{\partial (y')^2}\\
        \frac{\partial^2\phi}{\partial z^2}=\frac{\partial^2\phi}{\partial (z')^2}
    \end{cases}
\end{align}
If the wave equation is true in the original frame, then in the new frame, we get:
\begin{align}
    \frac{\partial^2\phi}{\partial x^2}+\frac{\partial^2\phi}{\partial y^2} + \frac{\partial^2\phi}{\partial z^2}-\frac{1}{c^2}\frac{\partial^2\phi}{\partial t^2}=0
\end{align}
\begin{align}
    \implies &\paran*{\gamma^2\beta^2\frac{\partial^2\phi}{\partial (ct')^2} + \gamma^2\frac{\partial^2\phi}{\partial (x')^2}-\gamma^2\beta\frac{\partial^2\phi}{\partial (ct')\partial (x')}} + \frac{\partial^2\phi}{\partial (y')^2} + \frac{\partial^2\phi}{\partial (z')^2} \\ 
    &- \paran*{\gamma^2\frac{\partial^2\phi}{\partial (ct')^2} + \gamma^2\beta^2\frac{\partial^2\phi}{\partial (x')^2} - \gamma^2\beta\frac{\partial^2\phi}{\partial(ct')\partial (x')}} = 0
\end{align}
\begin{align}
    &\implies \gamma^2(1-\beta^2)\frac{\partial^2\phi}{\partial (x')^2}+\frac{\partial^2\phi}{\partial (y')^2} + \frac{\partial^2\phi}{\partial (z')^2}  - \gamma^2(1-\beta^2)\frac{\partial^2\phi}{\partial (ct')^2}=0\\
    &\implies \frac{\partial^2\phi}{\partial (x')^2}+\frac{\partial^2\phi}{\partial (y')^2} + \frac{\partial^2\phi}{\partial (z')^2}  - \frac{1}{c^2}\frac{\partial^2\phi}{\partial (t')^2}=0
\end{align}
(Note: recall that $\gamma^2 = \frac{1}{1-\beta^2}$, and $\frac{\partial^2\phi}{\partial (ct')^2} = \frac{\partial}{\partial (ct')}\paran*{\frac{\partial\phi}{\partial (ct')}} = \frac{1}{c}\frac{\partial}{\partial t'}\paran*{\frac{1}{c}\frac{\partial\phi}{\partial t'}}=\frac{1}{c^2}\frac{\partial^2\phi}{\partial (t')^2}$).

Hence, the wave equation is still true under a Lorentz Boost.

\subsection*{3) Spatial Rotations:}
We want to generalize Lorentz Boost into arbitrary Lorentz Transformation (which is always a composition of spatial rotations and Lorentz Boost, if the orientation of the vectors remain the same). Hence, it suffices to check any spatial rotation matrix (i.e. only change the spatial basis, but not the time). More generally, if $U$ is an arbitrary unitary matrix of $\RR^{3\times 3}$ (which includes all rotation matrices), then the spatial unitary transformation is:
\begin{align}
    \begin{pmatrix}
        ct'\\x'\\y'\\z'
    \end{pmatrix}=\left(\begin{array}{c|c}
        1 & 0\\
        \hline
        0 & U
    \end{array}\right)\begin{pmatrix}
        ct\\x\\y\\z
    \end{pmatrix}
\end{align}
So, we get that $t' = t$, and $r^{\mu'} = u_{\mu'\mu}r^\mu$ (where $\mu'$ represents the basis $x',y',z'$ that has been transformed from $x,y,z$ respectively through $U$). Hence, we get the following equations for wave $\phi$:
\begin{align}
    &\frac{\partial\phi}{\partial r^\mu} = \frac{\partial\phi}{\partial r^{\mu'}}\frac{\partial r^{\mu'}}{\partial r^\mu} = u_{\mu'\mu}\frac{\partial\phi}{\partial r^{\mu'}}\\
    &\implies \frac{\partial^2\phi}{\partial (r^\mu)^2} = u_{\mu'\mu}u_{\nu'\mu}\frac{\partial^2\phi}{\partial r^{\mu'}\partial r^{\nu'}}
\end{align}
Hence, we get the following when summing up the partials of $x,y,z$:
\begin{align}
    \sum_{\mu \in \{x,y,z\}}\frac{\partial^2\phi}{\partial (r^\mu)^2} &= \sum_{\mu\in\{x,y,z\}}\sum_{\mu'\in\{x',y',z'\}}\sum_{\nu'\in\{x',y',z'\}}u_{\mu'\mu}u_{\nu'\mu}\frac{\partial^2\phi}{\partial r^{\mu'}\partial r^{\nu'}}\\
    &= \sum_{\mu\in\{x,y,z\}}\sum_{\mu'=\nu'}(u_{\mu'\mu})^2\frac{\partial^2\phi}{\partial (r^{\mu'})^2} + \sum_{\mu\in \{x,y,z\}}\sum_{\mu'\neq \nu'}u_{\mu'\mu}u_{\nu'\mu}\frac{\partial^2\phi}{\partial r^{\mu'}\partial r^{\nu'}}
\end{align}
Now, recall that $U$ is a unitary matrix, then it means the set of its row vectors forms an orthonormal basis. Hence, for the first sum above, if fixing $\mu'$ and sum over $\mu\in\{x,y,z\}$, we get $\sum_{\mu\in\{x,y,z\}}(u_{\mu'\mu})^2 = 1$ (since it is the norm square of $(u_{\mu'x},u_{\mu'y},u_{\mu'z})$, a row vector of $U$ which has norm $1$); then, for the second summation, if fixing $\mu', \nu'$ (where $\mu'\neq \nu'$), we get $\sum_{\mu\in\{x,y,z\}}u_{\mu'\mu}u_{\nu'\mu}=0$ (since this sum represents the dot product $(u_{\mu'x},u_{\mu'y},u_{\mu'z})\cdot (u_{\nu'x},u_{\nu'y},u_{\nu'z})$; with $\mu'\neq \nu'$, the two are distinct row vectors, then with orthogonality, the dot product is $0$). 

Then,the second sum above is identically $0$, while the first sum reduces to the sum over $\mu'$. So, the above becomes:
\begin{align}
    \sum_{\mu\in \{x,y,z\}}\frac{\partial^2\phi}{\partial (r^\mu)^2} = \sum_{\mu'\in\{x',y',z'\}}
\frac{\partial^2\phi}{\partial (r^{\mu'})^2}\end{align}
Which, with $t=t'$, we get the following equation:
\begin{align}
    &\paran*{\frac{\partial^2\phi}{\partial x^2}+\frac{\partial^2\phi}{\partial y^2}+\frac{\partial^2\phi}{\partial z^2}}-\frac{1}{c^2}\frac{\partial^2\phi}{\partial t^2} = 0 \iff \paran*{\frac{\partial^2\phi}{\partial(x')^2}+\frac{\partial^2\phi}{\partial (y')^2}+\frac{\partial^2\phi}{\partial (z')^2}}-\frac{1}{c^2}\frac{\partial^2\phi}{\partial (t')^2} = 0
\end{align}
Hence, the wave equation is also in the same form under any spatial unitary transformation (which in particular is true for spatial rotation).

\hfil

Since we've verified that under arbitrary translation of spacetime, Lorentz Boost, and Spatial Rotation, the wave equation all takes the same form, then under any Poincare Transformation (which is the composition of Lorentz Transformation and translation, while Lorentz Transformation is a composition of Lorentz Boost and Spatial rotation), the wave equation still maintains in the same form.

\end{document}