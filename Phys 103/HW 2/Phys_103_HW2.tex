\documentclass{article}
\usepackage[margin = 2.54cm]{geometry} % set margin to traditional doc

%packages
\usepackage{graphicx} % Required for inserting images
\usepackage[most]{tcolorbox} %for creating environments
\usepackage{amsmath}
\usepackage{amssymb}
\usepackage{mathtools}
\usepackage{verbatim}
\usepackage[utf8]{inputenc}
\usepackage[dvipsnames]{xcolor} %for importing multiple colors
\usepackage{hyperref} %for creating links to different sections

\linespread{1.2} %controlling line spread

%define colors i like
\definecolor{myTeal}{RGB}{0,128,128}
\definecolor{myGreen}{RGB}{34,170,34}
\definecolor{mySapphire}{RGB}{15,82,186}
\definecolor{myEmerald}{RGB}{50.4, 130, 90}

%create math environments, can add [section] or [subsection] to add index counter based on sections/subsections
\newtheorem{define}{Definition}
\newtheorem{prop}{Proposition}
\newtheorem{thm}{Theorem}
\newtheorem{question}{Question}
\newtheorem{lemma}{Lemma}

%setup colored box environment for each math env above
\tcolorboxenvironment{define}{
    enhanced, colframe=myTeal!50!teal, colback=myTeal!10,
    arc=5mm, lower separated=false, fonttitle=\bfseries
}
\tcolorboxenvironment{prop}{
    enhanced, colframe=myGreen!50!black, colback=myGreen!15,
    arc=5mm, lower separated=false, fonttitle=\bfseries
}
\tcolorboxenvironment{thm}{
    enhanced, colframe=mySapphire!50!mySapphire, colback=mySapphire!15,
    arc=5mm, lower separated=false, fonttitle=\bfseries
}
\tcolorboxenvironment{question}{
    enhanced, colframe=blue!50!black, colback=blue!10,
    arc=5mm, lower separated=false, fonttitle=\bfseries
}
\tcolorboxenvironment{lemma}{
    enhanced, colframe=myEmerald!50!myEmerald, colback=myEmerald!10,
    arc=5mm, lower separated=false, fonttitle=\bfseries
}

%setup hyperlink within pdf
\hypersetup{
    colorlinks=true,
    linkcolor=blue,
    filecolor=magenta,      
    urlcolor=cyan,
    pdftitle={Overleaf Example},
    pdfpagemode=FullScreen,
}

%common command (add to template)
%general
\newcommand{\FF}{\mathbb{F}}
\newcommand{\NN}{\mathbb{N}}
\newcommand{\ZZ}{\mathbb{Z}}
\newcommand{\QQ}{\mathbb{Q}}
\newcommand{\RR}{\mathbb{R}}
\newcommand{\CC}{\mathbb{C}}

\newcommand{\Id}{\textmd{Id}} %identity
\newcommand{\lcm}{\textmd{lcm}}
\DeclarePairedDelimiter{\abs}{\lvert}{\rvert}
\DeclarePairedDelimiter{\norm}{\lVert}{\rVert}
\DeclarePairedDelimiter{\paran}{(}{)}%paranthesis
\DeclarePairedDelimiter{\bracket}{\langle}{\rangle}

%algebra
\newcommand{\Gal}{\textmd{Gal}}
\newcommand{\Aut}{\textmd{Aut}}
\newcommand{\End}{\textmd{End}}
\newcommand{\Coker}{\textmd{Coker}}
\newcommand{\Hom}{\textmd{Hom}}
\newcommand{\Nil}{\textmd{Nil}}

%analysis
\newcommand{\Vol}{\textmd{Vol}}

%complex
\newcommand{\Real}{\textmd{Re}}
\newcommand{\Imag}{\textmd{Im}} %can also be used for Image
\newcommand{\Res}{\textmd{Res}}

%lie algebra
\newcommand{\gl}{\mathfrak{gl}}

\title{Phys 103 HW2}
\author{Zih-Yu Hsieh}

\begin{document}
\maketitle

\section{}
\begin{question}\label{q1}
    Consider an underdamped oscillator. Technically, because the amplitude decreases, the motion is not periodic and there is thus no period. We can, however, something that's enough like the period that we often just call it "the period".
    \begin{itemize}
        \item[(a)] If the oscillator starts at $x(0)=0$ (but with some initial velocity), show that subsequent zeroes are located at $t=\frac{n\pi}{w_1}$, $n \in \ZZ$. Defining the period as the amount of time to get two zeroes thus gives $\frac{2\pi}{w_1}$.
        \item[(b)] Againg setting $x(0)=0$, show that the local maxima in $x(t)$ do not occur at $t=\frac{(2n+1/2)\pi}{w_1}$; the maxima do not occur a quarter period after a zero, as would be the case for sinusoidal motion. On the other hand, show that the time between successive maxima is $\frac{2\pi}{w_1}$; defining the period as the amount of time between successive local maxima thus gives $\frac{2\pi}{w_1}$ as before. 
    \end{itemize}
\end{question}

\textbf{Pf:}

\subsection*{(a)}
Given the oscillator has the motion $x(t)=Ae^{-\beta t}\cos(w_1t-\phi)$, since $x(0) = A\cos(\phi)=0$, WLOG, can guess $\phi = \frac{\pi}{2}$ (if its $n\pi +\frac{\pi}{2}$, one can always change the constant $A$ accordingly). So, $x(t)=Ae^{-\beta t}\cos(w_1 t-\frac{\pi}{2})=Ae^{-\beta t}\sin(w_1 t)$ (here, assume $A>0$).

Which, if $x(t) = 0$, we have $Ae^{-\beta t}\sin(w_1 t)=0$. Since $A, e^{-\beta t}\neq 0$, we must have $\sin(w_1 t)=0$. As a result, $t = \frac{n\pi}{w_1}$ for $n \in \ZZ$. So, if the period is defined to be the amount of time to get two zeros, the period is $\frac{2\pi}{w_1}$ (since subsequent zeros apart each other with time $\frac{\pi}{w_1}$).

\subsection*{(b)}
If using the same equation $x(t)=Ae^{-\beta t}\sin(w_1t)$, to calculate where the local maxima is, we first consider up to its derivative:
\begin{equation}
    x'(t) = -A\beta e^{-\beta t}\sin(w_1 t)+Ae^{-\beta t}w_1\cos(w_1 t)
\end{equation}
Which, for any $n \in \ZZ$, if plug in $t=\frac{(2n+1/2)\pi}{w_1}$ to the derivative, we get:
\begin{align}
    x'\left(\frac{(2n+1/2)\pi}{w_1}\right) &= -Ae^{-\beta t}\left(\beta\sin\left(w_1\frac{(2n+1/2)\pi}{w_1}\right)-w_1\cos\left(w_1\frac{(2n+1/2)\pi}{w_1}\right)\right)\\
    &= -Ae^{-\beta t}\cdot \beta \neq 0
\end{align}
This shows that $t=\frac{(2n+1/2)\pi}{w_1}$ is no longer local maxima, since $x'(t)\neq 0$ at this point.

\hfil

However, if consider where $x'(t)=0$, suppose $t_0 \in [0,2\pi)$ satisfies $x'(t_0)=0$, we get the following relation:
\begin{align}
    x'(t_0)&=-Ae^{-\beta t_0}(-\beta\sin(w_1 t_0)+w_1\cos(w_1 t_0)) = 0,\quad Ae^{-\beta t_0}\neq 0\\
    &\implies -\beta\sin(w_1 t_0)+w_1\cos(w_1 t_0)=0\\
    &\implies \tan(w_1 t_0)=\frac{w_1}{\beta}
\end{align}
Hence, for any $n \in \ZZ$, $t=\frac{n\pi}{w_1}+t_0$ all satisfies $\tan(w_1 t)=\frac{w_1}{\beta}$,which are all zeros of $x'(t_0)$.

On the other hand, if consider the second derivative, we get the following:
\begin{equation}
    x''(t) = A(\beta^2-w_1^2)e^{-\beta t}\sin(w_1 t)-2A\beta w_1e^{-\beta t}\cos(w_1 t)
\end{equation}
Given that $t_0$ is local maximuj, $x''(t_0)<0$; however, for all $n\in\ZZ$, if $n$ is odd ($n=2k+1$ for some $k\in \ZZ$), the time $t=\frac{n\pi}{w_1}+t_0$ satisfies:
\begin{align}
    x''\left(\frac{n\pi}{w_1}+t_0\right)&=A(\beta^2-w_1^2)e^{-\beta t}\sin\left(w_1\frac{n\pi}{w_1}+w_1t_0\right) - 2A\beta w_1e^{-\beta t}\cos\left(w_1\frac{n\pi}{w_1}+w_1t_0\right)\\
    &= A(\beta^2-w_1^2)e^{-\beta (\frac{n\pi}{w_1}+t_0)}\sin((2k+1)\pi +w_1t_0)-2A\beta w_1e^{-\beta (\frac{n\pi}{w_1}+t_0)}\cos((2k+1)\pi +w_1t_0)\\
    &= -e^{-\beta\frac{n\pi}{w_1}}\left(A(\beta^2-w_1^2)e^{-\beta t_0}\sin(w_1t_0)-2A\beta w_1e^{-\beta t_0}\cos(w_1 t_0)\right)\\
    &= -e^{-\beta \frac{n\pi}{w_1}}\cdot x''(t_0)>0
\end{align}
Which, showing that $t=\frac{n\pi}{w_1}+t_0$ are all local minimum (since second derivatives are all positive). 

If $n$ is even instead ($n=2k$ for some $k\in\ZZ$), the above equation has no negative in the front, hence the second derivative remains negative, showing that $t=\frac{2k \pi}{w_1}+t_0$ are all local maxima.
So, any subsequent local maxima have a time difference of $\frac{2\pi}{w_1}$, showing that the period $\frac{2\pi}{w_1}$ can also be determined by the time between subsequent local maxima.

\break

\section{((b) not done)}
\begin{question}\label{q2}
    
    \hfil

    \begin{itemize}
        \item[(a)] Show that the energy $E$ of an underdamped oscillator (with $x=Ae^{-\beta t}\cos(w_1 t+\phi)$) is 
        $$E=\frac{1}{2}kA^2e^{-2\beta t}\left(1+\frac{1}{2Q}\cos\left(2w_1 t+2\phi-\arccos\frac{1}{2Q}\right)\right)$$
        \item[(b)] The overall exponential is easy to explain: the amplitude of the motion decreases as the damping dissipates energy, so the energy corresponding decays. Explain the physical origin of the cosine term.
        \item[(c)] $\frac{dE}{dt}$ tells us the rate at which the oscillator is losing energy. However, we can make a couople of improvements. First, rather than expressing the rate of energy loss as \emph{per unit time} we can also express it \emph{per oscillation}, as the oscillation itself provides a natural timescale. Similarly, rather than the absolute amount of energy lost, we can express the fractional energy loss by dividing by $E$.
        Next, the oscillatory terms are because of an actual physical effect: the energy does wobble a little throughout a cycle. If the oscillator is only weakly damped ($Q>>1$), we likely care much more about the overall decay after many cycles than about the slight wobble within each cycle. By time averaging over one period, we can do awway with the wobbles. These two together give us a dimensionless measure of how fast the oscillator loses on average. Show that
        $$-\left<\frac{1}{E}\frac{dE}{d\tilde{t}}\right>=\frac{2\pi}{Q}\left(1-\frac{1}{4Q^2}\right)^{-1/2}$$
        where $\tilde{t}=\frac{t}{2\pi/w_1}$ and the angle brackets indicate that we're averaging over a duration of $\frac{2\pi}{w_1}$. One interpretation of a "high quality" oscillator is the one that loses energy very slowly; if $Q$ is large, doubling the quality factor causes the oscillator to lose energy half as quickly.
    \end{itemize}
\end{question}

\textbf{Pf:}

\subsection*{(a)}
Recall that the elastic potential energy $U=\frac{1}{2}kx^2$, the kinetic energy is $K=\frac{1}{2}mv^2$, the natural frequency square $w_0^2=\frac{k}{m}$, and the frequency square $w_1^2=w_0^2-\beta^2$.

With $x(t)=Ae^{-\beta t}\cos(w_1t+\phi)$, we have the derivative $x'(t)=v(t)=-Ae^{-\beta t}(\beta \cos(w_1t+\phi)+w_1\sin(w_1t+\phi))$. Then plug in above, we get the energy as:
\begin{equation}
    U=\frac{1}{2}kx^2 = \frac{1}{2}kA^2e^{-2\beta t}\cos^2(w_1t+\phi)
\end{equation}
\begin{align}
    K&=\frac{1}{2}mv^2 = \frac{1}{2}mA^2e^{-2\beta t}(\beta^2\cos^2(w_1t+\phi)+w_1^2\sin^2(w_1t+\phi)+2\beta w_1\sin(w_1t+\phi)\cos(w_1t+\phi))\\
    &= \frac{1}{2}mA^2e^{-2\beta t}(w_0^2\sin^2(w_1t+\phi)+\beta^2(\cos^2(w_1t+\phi)-\sin^2(w_1t+\phi))+\beta w_1\sin(2w_1t+2\phi))\\
    &= \frac{1}{2}kA^2e^{-2\beta t}\sin^2(w_1t+\phi) + \frac{1}{2}mA^2e^{-2\beta t}(\beta^2\cos(2w_1t+2\phi)+\beta w_1\sin(2w_1t+2\phi))\\
    &=\frac{1}{2}kA^2e^{-2\beta t}\sin^2(w_1t+\phi) + \frac{1}{2}kA^2e^{-2\beta t}\cdot\frac{1}{w_0^2}\left(\beta^2\cos(2w_1t+2\phi)+\beta w_1\sin(2w_1t+2\phi)\right)
\end{align}
Recall that the linear combination of $\sin,\cos$ can be given as follow, for all $A,B\in \RR$:
\begin{equation}
    A\sin(x)+B\cos(x) = \sqrt{A^2+B^2}\cos\left(x-\arctan\left(\frac{B}{A}\right)\right)
\end{equation}
So, the kinetic energy can then be expressed as:
\begin{align}
    K&=\frac{1}{2}kA^2e^{-2\beta t}\sin^2(w_1t+\phi)+\frac{1}{2}kA^2e^{-2\beta t}\cdot\left(\frac{\beta}{w_0}\right)^2\cdot \sqrt{1+\left(\frac{w_1}{\beta}\right)^2}\cos\left(2w_1t+2\phi - \arctan\frac{w_1}{\beta}\right)\\
    &=\frac{1}{2}kA^2e^{-2\beta t}\sin^2(w_1t+\phi)+\frac{1}{2}kA^2e^{-2\beta t}\cdot\left(\frac{\beta}{w_0}\right)^2\cdot \sqrt{\frac{\beta^2+(w_0^2-\beta^2)}{\beta^2}}\cos\left(2w_1t+2\phi - \arctan\frac{\sqrt{w_0^2-\beta^2}}{\beta}\right)\\
    &= \frac{1}{2}kA^2e^{-2\beta t}\sin^2(w_1t+\phi)+\frac{1}{2}kA^2e^{-2\beta t}\cdot\left(\frac{\beta}{w_0}\right)^2\cdot \frac{w_0}{\beta}\cos\left(2w_1t+2\phi - \arccos\frac{\beta}{w_0}\right)
\end{align}
(Note: $\arctan(\sqrt{w_0^2-\beta^2}/\beta)=\arccos(\beta/w_0)$ can be derived through right triangle's side relations).

Then, the total energy is given as follow:
\begin{align}
    E = K+U &= \frac{1}{2}kA^3e^{-2\beta t}(\cos^2(w_1t+\phi)+\sin^2(w_1t+\phi)) + \frac{1}{2}kA^2e^{-2\beta t}\frac{2\beta}{2w_0}\cos\left(2w_1t+2\phi-\arccos\frac{2\beta}{2w_0}\right)\\
    &= \frac{1}{2}kA^2e^{-2\beta t}\left(1+\frac{1}{2Q}\cos\left(2w_1t+2\phi-\arccos\frac{1}{2Q}\right)\right)
\end{align}
(Note: recall that $Q=\frac{w_0}{2\beta}$, so $\frac{2\beta}{2w_0} = \frac{1}{2Q}$).

\subsection*{(b) (Not done)}

\subsection*{(c)}
Given that $\tilde{t}=\frac{t}{2\pi/w_1}$, then $\frac{dE}{d\tilde{t}} = \frac{dE}{dt}\frac{dt}{d\tilde{t}} = \frac{2\pi}{w_1}\frac{dE}{dt}$. Which, define $\phi_1 = 2\phi-\arccos\frac{1}{2Q}$, calculating the derivative, we get:
\begin{align}
    \frac{dE}{d\tilde{t}} &= \frac{2\pi}{w_1}\cdot \frac{1}{2}kA^2\left(-2\beta e^{-2\beta t}\left(1+\frac{1}{2Q}\cos\left(2w_1t+2\phi-\arccos\frac{1}{2Q}\right)\right) -\frac{w_1}{Q}e^{-2\beta t}\sin\left(2w_1t+2\phi-\arccos\frac{1}{2Q}\right)\right)\\
    &= -\frac{2\pi}{w_1}\cdot 2\beta E-\frac{2\pi}{w_1}\cdot \frac{w_1}{Q}\cdot\frac{1}{2}kA^2e^{-2\beta t}\sin\left(2w_1t+\phi_1\right)
\end{align}
Which, consider the term $-\frac{1}{E}\frac{dE}{d\tilde{t}}$, we get:
\begin{align}
    -\frac{1}{E}\frac{dE}{d\tilde{t}} &= \frac{2\pi}{w_1}\cdot 2\beta +\frac{2\pi}{w}\cdot \frac{w_1}{Q}\cdot\frac{\frac{1}{2}kA^2e^{-2\beta t}\sin(2w_1t+\phi_1)}{\frac{1}{2}kA^2e^{-2\beta t}\left(1+\frac{1}{2Q}\cos\left(2w_1t+\phi_1\right)\right)}\\
    &= \frac{2\pi}{w_1}\cdot 2\beta+\frac{2\pi}{w_1}\cdot\frac{2w_1}{2Q}\cdot\frac{\sin(2w_1t+\phi_1)}{1+\frac{1}{2Q}\cos(2w_1t+\phi_1)}
\end{align}
For the first term, taking the average over duration $\frac{2\pi}{w_1}$, we get $\frac{2\pi}{w_1}\cdot 2\beta$ (since it is a constant). For the second ter, since $u=1+\frac{1}{2Q}\cos(2w_1t+\phi_1)$ has derivative $-\frac{2w_1}{2Q}\sin(2w_1t+\phi_1)$, taking the average, we get:
\begin{align}
    \frac{1}{2\pi/w_1}\int_{t=0}^\frac{2\pi}{w_1}\frac{2\pi}{w_1}\cdot\frac{2w_1}{2Q}\cdot\frac{\sin(2w_1t+\phi_1)}{1+\frac{1}{2Q}\cos(2w_1t+\phi_1)} dt &= -\int_{t=0}^{\frac{2\pi}{w_1}}\frac{\frac{d}{dt}\left(1+\frac{1}{2Q}\cos(2w_1t+\phi_1)\right)}{1+\frac{1}{2Q}\cos(2w_1t+\phi_1)}dt\\
    &= -\ln\left(1+\frac{1}{2Q}\cos(2w_1t+\phi_1)\right)\bigg|_{0}^{\frac{2\pi}{w_1}}\\
    &= 0
\end{align}
(Note:recall that $\cos(2w_1t+\phi_1)$ has period $\frac{2\pi}{w_1}$).
 
As a result, since average can be distributed over addition, we get the following:
\begin{equation}
    -\left<\frac{1}{E}\frac{dE}{d\tilde{t}}\right>=\left<-\frac{1}{E}\frac{dE}{d\tilde{t}}\right> = \frac{2\pi}{w_1}\cdot 2\beta
\end{equation}
With $w_1 = \sqrt{w_0^2-\beta^2} = w_0\sqrt{1-\left(\frac{\beta}{w_0}\right)^2}$, and $Q=\frac{w_0}{2\beta}$, such average becomes:
\begin{equation}
    -\left<\frac{1}{E}\frac{dE}{d\tilde{t}}\right> = \frac{2\pi}{w_0\sqrt{1-\left(\frac{2\beta}{2w_0}\right)^2}}\cdot 2\beta  = \frac{2\pi}{Q}\left(1-\frac{1}{4Q^2}\right)^{-1/2}
\end{equation}

\break

\section{(Not done)}
\begin{question}\label{q3}
    Consider a damped oscillator with natural frequency $w_0$ and damping constant $\beta$, driven by a force $F_0\cos(wt)$.
    \begin{itemize}
        \item[(a)] Show that the average power delivered to the oscillator by the driving force is 
        $$\bracket*{P}=m\beta w^2A^2$$
        and that the average power dissipated by the damping force is also the same.
        \item[(b)] Find the driving frequency which maximizes the power, assuming $F_0, w_0, \beta$ are all fixed.
    \end{itemize}
\end{question}

\textbf{Pf:}

If the differential equation is $\ddot x+2\beta \dot x + w_0^2x = \frac{\tilde{F_0}}{m}e^{iwt}$ (where $\tilde{F_0}=F_0e^{i\delta}$ for phase $\delta$; since $\delta=0$ here, $\tilde{F_0}=F_0$; divide by $m$ is because the initial setup for force is $m\ddot x$, the whole differential equation is divided by $m$), then the solution is $\frac{F_0/m}{(w_0^2-w^2)+2i\beta w}e^{iwt} = \frac{F_0}{m\sqrt{(w_0^2-w^2)^2+(2\beta w)^2}}((w_0^2-w^2)-2\beta wi)(\cos(wt)+i\sin(wt))$. Which, taken the real part as the solution (which corresponds to the solution of $F_0\cos(wt)$), we get:
\begin{equation}
    x(t) = \frac{F_0}{m\sqrt{(w_0^2-w^2)^2+(2\beta w)^2}}\paran*{(w_0^2-w^2)\cos(wt)+2\beta w\sin(wt)}
\end{equation}
Which, this is the stable state solution of the driven oscillator, which we can use tfor calculation. (Note: based on the expression, since the linear combination of $B\sin(wt)+C\cos(wt)=A\cos(wt+\phi)$ has amplitude $A=\sqrt{B^2+C^2}$, then $x(t)$ in fact has amplitude $A=\frac{F_0}{m}$).

\subsection*{(a) (Not done)}
\subsubsection*{Power of Driving force:}
Under $1$-dimension, the power $P = Fv$ (where $v$ is the velocity), hence we first need to find the velocity:
\begin{equation}
    v(t)=x'(t) = \frac{F_0w}{m\sqrt{(w_0^2-w^2)^2+(2\beta w)^2}}(2\beta w\cos(wt)-(w_0^2-w^2)\sin(wt))
\end{equation}
Then, with driving force $F_{\textmd{drive}}(t)=F_0\cos(wt)$, the power is given by:
\begin{align}
    P_{\textmd{drive}}(t)&=F_{\textmd{drive}}(t)v(t) = \frac{F_0^2w}{m\sqrt{(w_0^2-w^2)^2+(2\beta w)^2}}(2\beta w\cos^2(wt)-(w_0^2-w^2)\sin(wt)\cos(wt))\\
    &=\frac{F_0^2w}{m\sqrt{(w_0^2-w^2)^2+(2\beta w)^2}}\paran*{2\beta w\cdot\frac{1+\cos(2wt)}{2}-\frac{w_0^2-w^2}{2}\sin(2wt)}
\end{align}
Which, notice that taking the average over period $\frac{2\pi}{w}$, since both $\cos(2wt),\sin(2wt)$ would provide an integral of $0$, the only term left is the constant (in the paranthesis, it's provided by $2\beta w \cdot \frac{1}{2}=\beta w$). Hence, the average driving power is:
\begin{equation}
    \bracket*{P_{\textmd{drive}}} = \frac{F_0^2\cdot\beta w^2}{m\sqrt{(w_0^2-w^2)^2+(2\beta w)^2}} = \frac{1}{\sqrt{(w_0^2-w^2)^2+(2\beta w)^2}}\cdot m\beta w^2A^2
\end{equation}

\subsubsection*{Power of Damping force:}
Recall that $\beta=\frac{b}{2m}$, and the damping force is provided by $F_{\textmd{damp}} = -bv$. Then, the power dissipated $P_{\textmd{damp}}=F_{\textmd{damp}}v = -bv^2$. Using the formula derived above, we get:
\begin{equation}
    P_{\textmd{damp}}(t)=-bv^2(t) = -b\cdot \frac{F_0^2w^2}{m^2((w_0^2-w^2)^2+(2\beta w)^2)}((2\beta w)^2\cos^2(wt)+(w_0^2-w^2)^2\sin^2(wt)-4\beta w(w_0^2-w^2)\sin(wt)\cos(wt))
\end{equation}
Which, recall that $\cos^2(wt)=\frac{1+\cos(2wt)}{2}$, $\sin^2(wt)=\frac{1-\cos(2wt)}{2}$, and $2\sin(wt)\cos(wt)=\sin(2wt)$. Taking the average over a time duration of $\frac{2\pi}{w}$, with integer multiples of $w$ being the frequency, $\sin(2wt),\cos(2wt)$ all provide $0$, then only the constant terms are left. Which, the average power dissipated by damping force is given by the constant in the above function:
\begin{align}
    \bracket*{P_\textmd{damp}} &= -b\cdot\frac{F_0^2w^2}{m^2((w_0^2-w^2)^2+(2\beta w)^2)}\paran*{\frac{(2\beta w)^2}{2}+\frac{(w_0^2-w^2)^2}{2}}\\
    &= -2m\beta\cdot \frac{F_0^2w^2}{2m^2} = -m\beta w^2\paran*{\frac{F_0}{w}}^2 = -m\beta w^2A^2
\end{align}
This shows that the two are in fact the same.


\subsection*{(b)}
Given that the average power is $m\beta w^2A^2$, since $\beta, w_0, F_0$ are fixed, 

\break


\section{}
\begin{question}\label{q4}
    Consider an oscillator driven by a sawtooth wave:
    $$f(t)=f_0\paran*{\frac{t}{T}-\left\lfloor\frac{1}{2}+\frac{t}{T}\right\rfloor}$$
    \begin{itemize}
        \item[(a)] Sketch a sawtooth wave. What is the period?
        \item[(b)] Calculate the Fourier series for a sawtooth wave. Plot a sawtooth wave and a Fourier series approximation on the same figure, keeping enough terms in the sum to get a reasonably good approximation.
        \item[(c)] Calculate the response of an oscillator (with natural frequency $w_0$ and damping parameter $\beta$) driven by a sawtooth wave. make plots for several different choices of the dimensionless parameters $Tw_0$ and $Q$, making sure to include examples that are both on and off resonance. Qualitatively describe your result.
    \end{itemize}
\end{question}

\textbf{Pf:}
\subsection*{(a)}
Based on the graph, we can conclude that the period is $T$ (also, for all $t \in \RR$, we have $f(t+T) = f_0\paran*{\frac{t+T}{T}-\left\lfloor\frac{1}{2}+\frac{t+T}{T}\right\rfloor}=f_0\paran*{1+\frac{t}{T}-\left\lfloor\frac{1}{2}+\frac{t}{T}+1\right\rfloor}$, with $\lfloor x+1\rfloor = \lfloor x\rfloor +1$, $f(t+1)=f(t)$).

\subsection*{(b)}
Within the interval $(-\frac{T}{2},\frac{T}{2})$, since $-\frac{1}{2}<\frac{t}{T}<\frac{1}{2} $, we have $0<\frac{1}{2}+\frac{t}{T}\rfloor<1$, hence the floor function constantly provides $0$. Therefore, $f(t) = f_0\frac{t}{T}$.

\break

\section{}
\begin{question}\label{q5}
\end{question}

\textbf{Pf:}

\break

\section{}
\begin{question}\label{q6}
\end{question}

\textbf{Pf:}

\break

\section{Extra Credit}
\begin{question}\label{q7}
\end{question}

\textbf{Pf:}

\break

\end{document}