\documentclass{article}
\usepackage[margin = 2.54cm]{geometry} % set margin to traditional doc

%packages
\usepackage{graphicx} % Required for inserting images
\usepackage[most]{tcolorbox} %for creating environments
\usepackage{amsmath}
\usepackage{amssymb}
\usepackage{mathtools}
\usepackage{verbatim}
\usepackage[utf8]{inputenc}
\usepackage[dvipsnames]{xcolor} %for importing multiple colors
\usepackage{hyperref} %for creating links to different sections

\linespread{1.2} %controlling line spread

%define colors i like
\definecolor{myTeal}{RGB}{0,128,128}
\definecolor{myGreen}{RGB}{34,170,34}
\definecolor{mySapphire}{RGB}{15,82,186}
\definecolor{myEmerald}{RGB}{50.4, 130, 90}

%create math environments, can add [section] or [subsection] to add index counter based on sections/subsections
\newtheorem{define}{Definition}
\newtheorem{prop}{Proposition}
\newtheorem{thm}{Theorem}
\newtheorem{question}{Question}
\newtheorem{lemma}{Lemma}

%setup colored box environment for each math env above
\tcolorboxenvironment{define}{
    enhanced, colframe=myTeal!50!teal, colback=myTeal!10,
    arc=5mm, lower separated=false, fonttitle=\bfseries, breakable
}
\tcolorboxenvironment{prop}{
    enhanced, colframe=myGreen!50!black, colback=myGreen!15,
    arc=5mm, lower separated=false, fonttitle=\bfseries, breakable
}
\tcolorboxenvironment{thm}{
    enhanced, colframe=mySapphire!50!mySapphire, colback=mySapphire!15,
    arc=5mm, lower separated=false, fonttitle=\bfseries, breakable
}
\tcolorboxenvironment{question}{
    enhanced, colframe=blue!50!black, colback=blue!10,
    arc=5mm, lower separated=false, fonttitle=\bfseries, breakable
}
\tcolorboxenvironment{lemma}{
    enhanced, colframe=myEmerald!50!myEmerald, colback=myEmerald!10,
    arc=5mm, lower separated=false, fonttitle=\bfseries, breakable
}

%setup hyperlink within pdf
\hypersetup{
    colorlinks=true,
    linkcolor=blue,
    filecolor=magenta,      
    urlcolor=cyan,
    pdftitle={Overleaf Example},
    pdfpagemode=FullScreen,
}

%common command (add to template)
%general
\newcommand{\FF}{\mathbb{F}}
\newcommand{\NN}{\mathbb{N}}
\newcommand{\ZZ}{\mathbb{Z}}
\newcommand{\QQ}{\mathbb{Q}}
\newcommand{\RR}{\mathbb{R}}
\newcommand{\CC}{\mathbb{C}}

\newcommand{\Id}{\textmd{Id}} %identity
\newcommand{\lcm}{\textmd{lcm}}
\DeclarePairedDelimiter{\abs}{\lvert}{\rvert}
\DeclarePairedDelimiter{\norm}{\lVert}{\rVert}
\DeclarePairedDelimiter{\paran}{(}{)}%paranthesis
\DeclarePairedDelimiter{\bracket}{\langle}{\rangle}

%algebra
\newcommand{\Gal}{\textmd{Gal}}
\newcommand{\Aut}{\textmd{Aut}}
\newcommand{\End}{\textmd{End}}
\newcommand{\Coker}{\textmd{Coker}}
\newcommand{\Hom}{\textmd{Hom}}
\newcommand{\Nil}{\textmd{Nil}}
\newcommand{\Char}{\textmd{char}}

%analysis
\newcommand{\Vol}{\textmd{Vol}}

%complex
\newcommand{\Real}{\textmd{Re}}
\newcommand{\Imag}{\textmd{Im}} %can also be used for Image
\newcommand{\Res}{\textmd{Res}}

%lie algebra
\newcommand{\gl}{\mathfrak{gl}}

%physics
\newcommand{\br}{\textbf{r}}
\newcommand{\bv}{\textbf{v}}
\newcommand{\ba}{\textbf{a}}

\title{Phys 103 HW2 Pass2}
\author{Zih-Yu Hsieh}

\begin{document}
\maketitle

\section*{Question 1}

\textbf{Pf:}
\subsection*{(a)}
In part (a), I misused the notation, where instead of $x(0)=A\cos(\phi)=0$, I should've written $0=x(0)=A\cos(\phi)$ (because we're given $x(0)=0$, not $A\cos(\phi)=0$).

\subsection*{(b)}
When doing the calculation, I forgot to assume that $t_0\in [0,\frac{2\pi}{w_1})$ is a local maximum of the function, but only assumed $x'(t_0)=0$ (which is not sufficient for local maximum).

\hfil

\rule{15.6cm}{0.1mm}

\hfil

\section*{Question 3}

\textbf{Pf:}

\subsection*{(a)}
In this section when calculating the power dissipated by the damping force, I should've use that the power dissipated $P_\textmd{dissipated} = -F_\textmd{damp} v = bv^2$ instead (or else it's phrasing that the power dissipated is negative, or the damping force is providing positive power).

\hfil

\rule{15.6cm}{0.1mm}

\hfil

\section*{Question 4}

\textbf{Pf:}
\subsection*{(a)}


\break

\section*{Question 6}

\textbf{Pf:}
\subsection*{(a)}
I forgot to answer the question whether it's going faster or slower eventually. This depends on the context:
\begin{itemize}
    \item If the oscillator is undamped, then the amplitude of the system is always the same. Hence, the period of the system stays constant, which it's not slower or faster.
    \item If the oscillator is damped, then its maximum amplitude is constantly decreasing, so when maximum amplitude is lower, with $T=\frac{2\pi}{w_0}\paran*{1+\frac{\phi_{\max}^2}{16}}$, we get that $T$ is decreasing. Hence, each cycle actually takes time less than $2$ seconds; Since each cycle takes less time, within the same time elapsed there are more cycles happened, hence it records a higher time elapsed than the actual time, showing that th pendulum eventually runs faster.
\end{itemize}


\end{document}