\documentclass{article}
\usepackage[margin = 2.54cm]{geometry} % set margin to traditional doc

%packages
\usepackage{graphicx} % Required for inserting images
\usepackage[most]{tcolorbox} %for creating environments
\usepackage{amsmath}
\usepackage{amssymb}
\usepackage{verbatim}
\usepackage[utf8]{inputenc}
\usepackage[dvipsnames]{xcolor} %for importing multiple colors
\usepackage{hyperref} %for creating links to different sections

\linespread{1.2} %controlling line spread

%define colors i like
\definecolor{myTeal}{RGB}{0,128,128}
\definecolor{myGreen}{RGB}{34,170,34}
\definecolor{mySapphire}{RGB}{15,82,186}
\definecolor{myEmerald}{RGB}{50.4, 130, 90}

%create math environments, can add [section] or [subsection] to add index counter based on sections/subsections
\newtheorem{define}{Definition}
\newtheorem{prop}{Proposition}
\newtheorem{thm}{Theorem}
\newtheorem{question}{Question}
\newtheorem{lemma}{Lemma}

%setup colored box environment for each math env above
\tcolorboxenvironment{define}{
    enhanced, colframe=myTeal!50!teal, colback=myTeal!10,
    arc=5mm, lower separated=false, fonttitle=\bfseries
}
\tcolorboxenvironment{prop}{
    enhanced, colframe=myGreen!50!black, colback=myGreen!15,
    arc=5mm, lower separated=false, fonttitle=\bfseries
}
\tcolorboxenvironment{thm}{
    enhanced, colframe=mySapphire!50!mySapphire, colback=mySapphire!15,
    arc=5mm, lower separated=false, fonttitle=\bfseries
}
\tcolorboxenvironment{question}{
    enhanced, colframe=blue!50!black, colback=blue!10,
    arc=5mm, lower separated=false, fonttitle=\bfseries
}
\tcolorboxenvironment{lemma}{
    enhanced, colframe=myEmerald!50!myEmerald, colback=myEmerald!10,
    arc=5mm, lower separated=false, fonttitle=\bfseries
}

%setup hyperlink within pdf
\hypersetup{
    colorlinks=true,
    linkcolor=blue,
    filecolor=magenta,      
    urlcolor=cyan,
    pdftitle={Overleaf Example},
    pdfpagemode=FullScreen,
}

%common command (add to template)
%general
\newcommand{\FF}{\mathbb{F}}
\newcommand{\NN}{\mathbb{N}}
\newcommand{\ZZ}{\mathbb{Z}}
\newcommand{\QQ}{\mathbb{Q}}
\newcommand{\RR}{\mathbb{R}}
\newcommand{\CC}{\mathbb{C}}

\newcommand{\Id}{\textmd{Id}} %identity
\newcommand{\lcm}{\textmd{lcm}}

%algebra
\newcommand{\Gal}{\textmd{Gal}}
\newcommand{\Aut}{\textmd{Aut}}
\newcommand{\End}{\textmd{End}}
\newcommand{\Coker}{\textmd{Coker}}
\newcommand{\Hom}{\textmd{Hom}}

%analysis
\newcommand{\Vol}{\textmd{Vol}}

%complex
\newcommand{\Real}{\textmd{Re}}
\newcommand{\Imag}{\textmd{Im}} %can also be used for Image
\newcommand{\Res}{\textmd{Res}}

%lie algebra
\newcommand{\gl}{\mathfrak{gl}}

%physics
\newcommand{\br}{\overline{r}}
\newcommand{\bv}{\overline{v}}
\newcommand{\ba}{\overline{a}}
\newcommand{\bzero}{\overline{0}}
\newcommand{\bF}{\overline{F}}

\title{Phys 103 HW1}
\author{Zih-Yu Hsieh}

\begin{document}
\maketitle

\section{}%1
\begin{question}\label{q1}
    Consider a puck sliding across a frictionless merry-go-round at speed $v$ on a trajectory which passes through the center. The merry-go-round rotates at angular velocity $w$ and has radius $R$.
    \begin{itemize}
        \item[(a)] Write down the $r,\phi$ coordinates of the puck as functions of time, as observed by an observer standing next to the merry-go-round. Take $t=0$ to be the time when the puck is at the edge of the merry-go-round, the spatial origin to be at the center of the merry-go-round, and $\phi=0$ to be the initial angular location of the puck. Is the observer in an inertial frame?
        \item[(b)] Write down the $r', \phi'$ coordinates of the puck as functions of time, as observed by an observer sitting on the edge of the merry-go-round. Take $t=0$ to be the time when the puck is at the edge of the merry-go-round, the spatial origin to be at the center of the merry-go-round, and $\phi'=0$ to be the initial angular location of the puck. Is the observer in an inertial frame?
    \end{itemize}
\end{question}

\textbf{Pf:}

For both parts, we'll assume that the codomain of $r,\phi$ is $\RR$ (i.e. allows negative radius $r$ for simplicity).
\subsection*{(a)}
First, we'll consider the distance of the puck away from the origin: On the $1$-dimensional trajectory, assume the origin (center) is position $r=0$, and at $t=0$, the puck is at the edge, which has position $r=R$ (assume that the puck initially has positive value position in the coordinate, so the distance from the center to the edge is the radius). Assume the puck is traveling in radial direction, relative to
 the track its velocity is given by $v$, hence the position $r(t) = \int vdt = vt+C$. With, $r(0) = 0+C = R$, we have $r(t)=R+vt$. This records the radial distance of the puck from the origin.

Now, we'll consider the angle of the puck relative to the observer (which we'll keep track of the angle of its initial position on the trajectory): With initial angle $\phi(0) = 0$, and angular velocity $w$, then the angle $\phi(t) = \int w dt = wt + C$. Where with $\phi(0) = 0+C = 0$, we have $\phi(t) = wt$.

\hfil

With $(r,\phi)$ being the given polar coordinates of the puck, in cartesian coordinates, the position is given by:
$$\br(t)=(r\cos(\phi),r\sin(\phi)) = (R+vt)(\cos(wt),\sin(wt))$$
Taking derivative, we get the velocity as follow:
$$\bv(t) = (v\cos(wt)-w(R+vt)\sin(wt), v\sin(wt)+w(R+vt)\cos(wt))$$
Taking the second derivative, we get the acceleration as follow:
$$\ba(t) = (-wv\sin(wt)- w(v\sin(wt) + w(R+vt)\cos(wt)), wv\cos(wt) + w(v\cos(wt)-w(R+vt)\sin(wt)))$$
$$ = (-2wv\sin(wt) - w^2(R+vt)\cos(wt), 2wv\cos(wt)-w^2(R+vt)\sin(wt))$$
Which, plug in $t=0$, we get $\ba(0) = (-w^2R, 2wv) \neq \bar{0}$, showing that there are time where the puck is under acceleration. If we assume the puck is an isolated object, then because for the observer the puck is accelerating, the observer is not in an inertial frame.

\subsection*{(b)}
When considering the distance of the puck away from the origin, using similar coordinates from part (a), assume the center has radius $r=0$, the puck initially has radius $r=R$, and the puck has speed $v$ in radial direction, then the position of the puck is again given as $r(t)=R+vt$ (the same derivation).

In contrast, for the angle $\phi$, since the observer is now on the edge of the merry-go-round, then the relative angle in between the observer and the puck's initial position never changes. So, $\phi(t)=0$ (where the initial angle of the puck is).

\hfil

With $(r,\phi)$ given as the polwer coordinates of the puck, in cartesian coordinates, the position is given by:
$$\br(t) = (r\cos(\phi), r\sin(\phi)) = (R+vt)(1,0) = (R+vt,0)$$
Hence, taken second derivative, we get the acceleration as follow:
$$\ba(t) = \bzero$$
If treating the puck as an isolated object again, then this observer is in an inertial frame (since the isolated object is not accelerating).

\break

\section{}%2
\begin{question}\label{q2}
    Consider an athlete putting a shot. Naturally, she would like to maximize the distance the shot travels. If the shot is put from a height $h$ and with initial speed $v_0$, what launch angle maximizes the distance traveled? Assume that $v_0$ and $h$ are independent of the launch angle, and ignore air resistance (but include gravity, if it wasn't obvious). You can (and should) assume that $h$ is "small" and give an answer to leading non-trivial order, but you should specify what counts as "small".
\end{question}

\textbf{Pf:}

We'll assume the vertical direction upward is the $\hat{y}$ axis, while the horizontal direction (where the shot is heading horizontally) is the $\hat{x}$ axis. We'll also assume the initial position of the shot $\br(0) = (0,h)$, and let $\theta$ represents the initial launching angle.

Since the only force taking into account is gravity, then the net force acting on the launch is $\bF_{net} = \bF_g = -mg\hat{y}$, hence the acceleration is given by $\ba = \frac{1}{m}\bF_{net} = -g\hat{y}$ (here assumes the mass of the shot is constant).

Taking integration, we get $\bv(t) = \int \ba dt = -gt\hat{y}+\overline{C}$ for some $\overline{C} \in \RR^2$. Which, recall that initial velocity has speed $v_0$ and launch angle $\theta$, hence $\bv(0) = (v_0\cos(\theta),v_0\sin(\theta)) = 0\cdot \hat{y}+\overline{C}$. Hence, $\bv(t) = (v_0\cos(\theta), v_0\sin(\theta)-gt)$.

Then, taking integration again, we get $\br(t) = \int \bv dt = (v_0\cos(\theta)t + r_x, v_0\sin(\theta)t - \frac{1}{2}gt^2 + r_y)$ for some $(r_x,r_y)\in \RR^2$, and with the assumption that $\br(0) = (0,h) = (0+r_x, 0-0+r_y)$, we get the following for position:
$$\br(t) = \left(v_0\cos(\theta)t, v_0\sin(\theta)t-\frac{1}{2}gt^2+h\right)$$

\hfil

Now, the objective is to solve for the horizontal distance the shot would travel, which we need to calculate the time it lands (i.e. the second coordinate of $\br(t)$ is 0). For $-\frac{1}{2}gt^2-v_0\sin(\theta)t+h = 0$, using quadratic formula, we get the following expression for $t$:
$$t = \frac{-v_0\sin(\theta)\pm \sqrt{v_0^2\sin^2(\theta)-4\cdot \left(-\frac{g}{2}\right)h}}{2\left(-\frac{g}{2}\right)} = \frac{v_0\sin(\theta)\pm\sqrt{v_0^2\sin^2(\theta)+2gh}}{g}$$
Since we can assume both $g,h>0$ (it's not reasonable to launch a shot from height $h<0$ in general), hence $(v_0^2\sin^2(\theta)+2gh)\geq v_0^2\sin^2(\theta)$, showing that $\sqrt{v_0^2\sin^2(\theta)+2gh} \geq |v_0\sin(\theta)|\geq v_0\sin(\theta)$. So, for the time $t>0$, we need positive sign for the square root, getting that the landing time is as follow:
$$t=\frac{v_0\sin(\theta)+\sqrt{v_0^2\sin^2(\theta)+2gh}}{g}$$
Since the launch starts at time $=0$, then the above time is exactly the time where the shot is in the air. Hence, the horizontal distance traveled is given by:
$$\Delta x=v_0\cos(\theta)t = v_0\cos(\theta)\cdot\frac{v_0\sin(\theta)+\sqrt{v_0^2\sin^2(\theta)+2gh}}{g} = v_0^2\sin(\theta)\cos(\theta)\frac{1+\sqrt{1+\frac{2gh}{v_0^2\sin^2(\theta)}}}{g}$$
This expression is relatively hard to solve for the maximum value with $\theta$, so using Taylor polynomial to approximate for the square root term is necessary.

Assume $\theta\in [k,\frac{\pi}{2}]$ for some $k>0$, or else the $\sin(\theta)$ in the denominator could go unbounded, if assume $h$ is small, can also assume $\frac{2gh}{v_0^2\sin^2(\theta)}$ is small, then the taylor polynomial of $\sqrt{1+x}$ could yield a good approximation. For $x$ close to $0$, up to linear order, its taylor polynomial is given as follow:
$$\sqrt{1+x}\approx 1+\frac{1}{2}x$$
Using this approximation for $x=\frac{2gh}{v_0^2\sin^2(\theta)}$, the horizontal displacement is approximated as follow:
$$\Delta x\approx v_0^2\sin(\theta)\cos(\theta)\cdot\frac{1+\left(1+\frac{1}{2}\cdot\frac{2gh}{v_0^2\sin^2(\theta)}\right)}{g} = \frac{2v_0^2}{g}\sin(\theta)\cos(\theta)+h\cdot\frac{\cos(\theta)}{\sin(\theta)} = \frac{2v_0^2}{g}\sin(\theta)\cos(\theta)+h\cot(\theta)$$
To get a local maximum, differentiate the above term with respect to $\theta$, we get:
$$\frac{d}{d\theta}\Delta x=\frac{2v_0^2}{g}(\cos^2(\theta)-\sin^2(\theta))-h\csc^2(\theta)=\frac{2v_0^2}{g}(1-2\sin^2(\theta))-\frac{h}{\sin^2(\theta)}$$
With the above expression being $0$, we get:
$$\frac{2v_0^2}{g}(1-2\sin^2(\theta))-\frac{h}{\sin^2(\theta)}=0 \implies (2\sin^2(\theta)-1)\sin^2(\theta)+\frac{gh}{2v_0^2}=0$$
With $x:=\sin^2(\theta)$, we get:
$$2x^2-x+\frac{gh}{2v_0^2}=0 \implies x=\sin^2(\theta)=\frac{1\pm\sqrt{1-\frac{4gh}{v_0^2}}}{4}$$
With the assumption that $\theta \geq k>0$, and $h$ being extremely small, then $\sqrt{1-\frac{4gh}{v_0^2}}$ would be pretty close to $1$, hence the negative sign would provide that $\sin^2(\theta)$ is approximately $0$, so we'll take the positive sign.

Finally, this indicates the angle $\theta$ for maximum launch distance is given as follow:
$$\sin^2(\theta)=\frac{1}{4}\left(1+\sqrt{1-\frac{4gh}{v_0^2}}\right),\quad \sin(\theta)=\frac{1}{2}\sqrt{1+\sqrt{1-\frac{4gh}{v_0^2}}}, \theta = \arcsin\left(\frac{1}{2}\sqrt{1+\sqrt{1-\frac{4gh}{v_0^2}}}\right)$$

\break

\section{}%3
\begin{question}\label{q3}
    Consider an object, such as a car, with mass $m$, starting from rest, moving in one dimension, and subject to two forces. One, provided by an engine with a certain power output, is $F_e = P/v$. The other is some quadratic drag from air resistance, $F_d = -cv^2$.$P,c$ are both positive constants. Unfortunately, it is not possible to analytically solve for the car's velocity as a function of time. However, we can get approximate analytic expressions which are very accurate.
    \begin{itemize}
        \item[(a)] First, nondimensionalize your equation, defining appropriate dimensionless time and velocity variables.
        \item[(b)] At the start of the motion, $\tilde{v}<< 1$, so air resistance is a small correction; the car's motion is essentially just acceleration driven by the engine. By ignoring the force due to air resistance, determine an (approximate) formula $\tilde{v}(\tilde{t})$, valid for $\tilde{v}<<1$.
        \item[(c)]Eventually, the car approaches terminal velocity. We can define 
        $$\epsilon = \tilde{v}_t-\tilde{v}$$
        If the car is going close to terminal velocity, $\epsilon <<1$. Expand the forces to linear order in $\epsilon$, and solve to get an (approximate) formula $\epsilon(\tilde{t})$, valid for $\epsilon<<1$.
        \item[(d)] Now comes the leap of faith: assume that either the velocity is small enough to ignore air resistance or the velocity is close to terminal velocity. In other words, we'll write $\tilde{v}(\tilde{t})$ as a piecewise function; before some time $\tilde{t}_*$ it takes the form of your solution to (c). Choose an appropriate switching time $\tilde{t}_*$ and write down such a piecewise function for $\tilde{v}(\tilde{t})$. You should fix the two undetermined constants by using that (i) the car starts at rest and (ii) $\tilde{v}$ should be continuous.Plot the piecewise function $\tilde{v}(\tilde{t})$.
        \item[(e)] Put all the constants back in to get $v(t)$. At what time $t$ does the car obtain $99\%$ of terminal velocity? You should give an analytic expression (using your approximate $\tilde{v}$). 
    \end{itemize}
\end{question}

\textbf{Pf:}
\subsection*{(a)}
Given that force $F=m\frac{dv}{dt}=F_e + F_d$, we get the following differential equation:
$$m\frac{dv}{dt}=\frac{P}{v}-cv^2$$
For some non-determined constant velocity $v_0$ and time $t_0$, let $\tilde{v}=\frac{v}{v_0}$, $\tilde{t}=\frac{t}{t_0}$, then we get the following expression:
$$\frac{d\tilde{v}}{d\tilde{t}} = \frac{d}{d\tilde{t}}\left(\frac{v(t)}{v_0}\right) = \frac{1}{v_0}\cdot \frac{dv}{dt}\cdot \frac{dt}{d \tilde{t}} = \frac{t_0}{v_0}\frac{dv}{dt}$$
Substituting all the expressions of $v$ and $t$, we get:
$$m\cdot \frac{v_0}{t_0}\frac{d\tilde{v}}{d\tilde{t}} = \frac{P}{v_0\tilde{v}}-c(v_0\tilde{v})^2$$


\subsection*{(b)}

\subsection*{(c)}

\subsection*{(d)}

\subsection*{(e)}

\break

\section{}%4
\begin{question}\label{q4}
    Consider a circular hoop (mass $M$ and radius $R$) standing vertically on the ground. Onto the hoop are threaded two beads (each mass $m$ and radius $r$), which move without friction along the hoop. The two beads start at rest practically at the top of the hoop (which requires $r<<R$; you can and should approximate for this being the case). One bead goes counterclockwise and the other clockwise.
    \begin{itemize}
        \item[(a)] What is the smallest value of $m/M$ for which the hoop rises off the ground?
        \item[(b)] Assuming the hoop does not rise off the ground, approximately how long does it take the beads to reach the bottom of the hoop? 
    \end{itemize}
\end{question}

\textbf{Pf:}

Since the beads appear on both sides, WLOG, can assume the behavior is symmetric, hence we only need to analyze the bead on the right side.
\subsection*{(a)}
In case for the beads to move on the circular hoop, they require centripetal force (given by $F_c = \frac{mv^2}{r}$, where $m,v$ are mass and speed of each individual bead, and $r$ is the radius, in this case $r=R$).

\textbf{Insert images}

If consider the force diagram, for beads at the bottom half of the hoop, the force applied on the hoop by the beads are directed downward, hence it's not possible for the hoop to rise off the ground. Then, if the hoop rises, it must happen when the beads are on the upper half of the hoop, so the angle given above $\theta \in [0,\frac{\pi}{2}]$.

\hfil

First, for each angle $\theta$, the position of the bead is parametrized by:
$$\br(\theta)=(R\sin(\theta), R+R\cos(\theta))$$
Compared to the top (where height is $2R$), at angle $\theta$, the height drop is given by $2R-(R+R\cos(\theta)) = R(1-\cos(\theta))$, hence the lost potential energy of the bead is $mgR(1-\cos(\theta))$. If assume all the potential energy is transferred to kinetic energy, then $\frac{1}{2}mv^2 = mgR(1-\cos(\theta))$, hence at angle $\theta$, the speed of the bead satisfies:
$$mv^2 = 2mgR(1-\cos(\theta))$$
Then, the centripetal force $F_c = \frac{mv^2}{R} = 2mg(1-\cos(\theta))$. With the force diagram, the centripetal force at angle $\theta$ is given as the following vector:
$$\bF_c = -F_c(\sin(\theta),\cos(\theta)) = -2mg(1-\cos(\theta))(\sin(\theta),\cos(\theta))$$

\hfil

Then, let $\bF_b$ be the force applied on the bead by the hoop, and $\bF_g = (0,-mg)$ is the gravity force on the bead. Then, for the motion of the bead to stay in the circular hoop, we need $\bF_g+\bF_b = \bF_c$ (the net force being the centripetal force), hence:
$$\bF_b = \bF_c - \bF_g = -2mg(1-\cos(\theta))(\sin(\theta),\cos(\theta)) - (0,-mg)$$
$$ = \left(-2mg(1-\cos(\theta))\sin(\theta), mg-2mg(1-\cos(\theta))\cos(\theta)\right)$$
Which, the force applied on the hoop by the bead (using Newton's Third Law) is given by:
$$\bF_h = -\bF_b = \left(2mg(1-\cos(\theta))\sin(\theta), 2mg(1-\cos(\theta))\cos(\theta)-mg\right)$$
So, the vertical component of $\bF_h$ is given by $(2mg(1-\cos(\theta))\cos(\theta)-mg)\hat{y}$; and since the two beads on both sides apply the vertical components in the same way, the total vertical force applied on the hoop by the beads is $\bF_h^{(y)} = 2(2mg(1-\cos(\theta))\cos(\theta)-mg)\hat{y}$.

\hfil

Finally, for the hoop to rise, the net vertical force on it must be directed upward. Since the only vertical forces on the hoop are $\bF_h^{(y)}$ and gravity force $-Mg\hat{y}$, the net vertical force on the hoop is:
$$\bF_h^{(y)}-Mg\hat{y} = \left(mg(-4\cos^2(\theta)+4\cos(\theta)-2)-Mg\right)\hat{y}$$
However, if calculate the maximum of the term $-4\cos^2(\theta)+4\cos(\theta)-2$ in the domain $\theta\in [0,\frac{\pi}{2}]$, first taking the derivative being $0$ we get:
$$8\cos(\theta)\sin(\theta)-4\sin(\theta) = 0,\quad 4\sin(\theta)(2\cos(\theta)-1)=0$$
Which, $\sin(\theta)=0$ (where $\theta=0$), or $\cos(\theta)=\frac{1}{2}$ must be satisfied.

So, the absolute maximum of this expression must occur at $\theta=0,\ \theta=\frac{\pi}{2}$, or $\cos(\theta)=\frac{1}{2}$. However, plugging into the expression, we get:
$$\theta=0\implies -4\cos^2(\theta)+4\cos(\theta)-2 = -4+4-2 = -2$$
$$\theta=\frac{\pi}{2}\implies -4\cos^2(\theta)+4\cos(\theta)-2=-4\cdot 0+4\cdot 0-2 = -2$$
$$\cos(\theta)=\frac{1}{2}\implies -4\cos^2(\theta)+4\cos(\theta)-2 = -4\cdot\frac{1}{4}+4\cdot\frac{1}{2}-2 = -1$$
Since all the possible maximum values are negative, that expression is negative.

However, since $m,M,g$ are all positive, the expression $mg(-4\cos^2(\theta)+4\cos(\theta)-2)-Mg<0$ (since both $mg, Mg$ are multiplied with a negative value regardless of $\theta$), hence the vertical forces applied on the hoop is always directed downward, showing that it's not possible for the hoop to rise off the ground.

\subsection*{(b)}


\break

\section{}%5
\begin{question}\label{q5}
    On the mastery questions, you showed that a rocket launched under the influence of gravity has a lower velocity, and that the "penalty" imposed by gravity is larger the longer it takes the rocket to burn its fuel. This suggests that, if launching to orbit off Earth, it would be best to make $dm/dt$ as large as possible. However, there's a technical challenge: burning fuel more rapidly generally requires a bigger engine, and the increasing mass of the engine imposes its own penalty. Let's take the rocket as having four components: fuel, fuel tank, engine, and payload initially, then 
    $$m_0 = m_f + m_t+m_e+m_p$$
    Let's assume that the rate of fuel consumption is proportional to the mass of the engine, 
    $$\alpha \equiv \left|\frac{dm}{td}\right|=m_e/\tau$$
    and that the effective exhaust speed is fixed at $u$, regardless of the size of the engine. $\tau$ is some constant with units of time (it's the amount of time for the engine to consume a mass of fuel equal to its own mass), and for any particular rocket, $\alpha$ will be constnat.
    \begin{itemize}
        \item[(a)] Calculate the velocity of the racket after it has consumed all its fuel. Your answer should depend on $u,g,\tau$ and the masses of the four components.
        \item[(b)] Your final velocity should depend only on mass ratios, not the absolute masses. Explain physically why doubling the mass of everything has no effect on the final velocity.
        \item[(c)]Since only mass ratios are important, we can define $\tilde{m_i} = m_i/m_f$ (Remember that $f$ stands for fuel, not final). $\tilde{m_p}$ is fixed by a combination of the desired payload and how large a rocket we can build; $\tilde{m_t}$ is more or less fixed by choice of fuel. $\tilde{m_e}$ is the variable we have control over. Find the $\tilde{m_e}$ which maximizes the final speed as a function of $u,g,\tau,\tilde{m_p},\tilde{m_t}$.
        \item[(d)]For a chemical rocket, $u\sim 3 \textmd{km/s}$, $\tau\sim 7\textmd{s}$, and $\tilde{m_t}\sim 1/10$. Plug in numbers to get the optimal $\tilde{m_e}$ for both a negligible payload $(\tilde{m_p}=0$) and a modest payload ($\tilde{m_p}=0.1$). For both cases, also calculate the initial acceleration of the rocket, expressed as a number of $g$'s. 
    \end{itemize}
\end{question}

\textbf{Pf:}

\break

\section{}%6
\begin{question}\label{q6}
\end{question}

\textbf{Pf:}

\end{document}