\documentclass{article}
\usepackage[margin = 2.54cm]{geometry} % set margin to traditional doc

%packages
\usepackage{graphicx} % Required for inserting images
\usepackage[most]{tcolorbox} %for creating environments
\usepackage{amsmath}
\usepackage{amssymb}
\usepackage{mathtools}
\usepackage{verbatim}
\usepackage[utf8]{inputenc}
\usepackage[dvipsnames]{xcolor} %for importing multiple colors
\usepackage{hyperref} %for creating links to different sections

\linespread{1.2} %controlling line spread

%define colors i like
\definecolor{myTeal}{RGB}{0,128,128}
\definecolor{myGreen}{RGB}{34,170,34}
\definecolor{mySapphire}{RGB}{15,82,186}
\definecolor{myEmerald}{RGB}{50.4, 130, 90}

%create math environments, can add [section] or [subsection] to add index counter based on sections/subsections
\newtheorem{define}{Definition}
\newtheorem{prop}{Proposition}
\newtheorem{thm}{Theorem}
\newtheorem{question}{Question}
\newtheorem{lemma}{Lemma}

%setup colored box environment for each math env above
\tcolorboxenvironment{define}{
    enhanced, colframe=myTeal!50!teal, colback=myTeal!10,
    arc=5mm, lower separated=false, fonttitle=\bfseries, breakable
}
\tcolorboxenvironment{prop}{
    enhanced, colframe=myGreen!50!black, colback=myGreen!15,
    arc=5mm, lower separated=false, fonttitle=\bfseries, breakable
}
\tcolorboxenvironment{thm}{
    enhanced, colframe=mySapphire!50!mySapphire, colback=mySapphire!15,
    arc=5mm, lower separated=false, fonttitle=\bfseries, breakable
}
\tcolorboxenvironment{question}{
    enhanced, colframe=blue!50!black, colback=blue!10,
    arc=5mm, lower separated=false, fonttitle=\bfseries, breakable
}
\tcolorboxenvironment{lemma}{
    enhanced, colframe=myEmerald!50!myEmerald, colback=myEmerald!10,
    arc=5mm, lower separated=false, fonttitle=\bfseries, breakable
}

%setup hyperlink within pdf
\hypersetup{
    colorlinks=true,
    linkcolor=blue,
    filecolor=magenta,      
    urlcolor=cyan,
    pdftitle={Overleaf Example},
    pdfpagemode=FullScreen,
}

%common command (add to template)
%general
\newcommand{\FF}{\mathbb{F}}
\newcommand{\NN}{\mathbb{N}}
\newcommand{\ZZ}{\mathbb{Z}}
\newcommand{\QQ}{\mathbb{Q}}
\newcommand{\RR}{\mathbb{R}}
\newcommand{\CC}{\mathbb{C}}

\newcommand{\Id}{\textmd{Id}} %identity
\newcommand{\lcm}{\textmd{lcm}}
\DeclarePairedDelimiter{\abs}{\lvert}{\rvert}
\DeclarePairedDelimiter{\norm}{\lVert}{\rVert}
\DeclarePairedDelimiter{\paran}{(}{)}%paranthesis
\DeclarePairedDelimiter{\bracket}{\langle}{\rangle}

%algebra
\newcommand{\Gal}{\textmd{Gal}}
\newcommand{\Aut}{\textmd{Aut}}
\newcommand{\End}{\textmd{End}}
\newcommand{\Coker}{\textmd{Coker}}
\newcommand{\Hom}{\textmd{Hom}}
\newcommand{\Nil}{\textmd{Nil}}

%analysis
\newcommand{\Vol}{\textmd{Vol}}

%complex
\newcommand{\Real}{\textmd{Re}}
\newcommand{\Imag}{\textmd{Im}} %can also be used for Image
\newcommand{\Res}{\textmd{Res}}

%lie algebra
\newcommand{\gl}{\mathfrak{gl}}

\title{Latex Template}
\author{Zih-Yu Hsieh}

\begin{document}
\maketitle

\section*{1}
\begin{question}\label{q1}
    Consider a puck sliding across a frictionless merry-go-round at speed $v$ on a trajectory which passes through the center. The merry-go-round rotates at angular velocity $w$ and has radius $R$.
    \begin{itemize}
        \item[(a)] Write down the $r,\phi$ coordinates of the puck as functions of time, as observed by an observer standing next to the merry-go-round. Take $t=0$ to be the time when the puck is at the edge of the merry-go-round, the spatial origin to be at the center of the merry-go-round, and $\phi=0$ to be the initial angular location of the puck. Is the observer in an inertial frame?
        \item[(b)] Write down the $r', \phi'$ coordinates of the puck as functions of time, as observed by an observer sitting on the edge of the merry-go-round. Take $t=0$ to be the time when the puck is at the edge of the merry-go-round, the spatial origin to be at the center of the merry-go-round, and $\phi'=0$ to be the initial angular location of the puck. Is the observer in an inertial frame?
    \end{itemize}
\end{question}

\textbf{Pf:}

Fxxk you.

\break

\section*{2}
\begin{question}\label{q2}
    Consider an athlete putting a shot. Naturally, she would like to maximize the distance the shot travels. If the shot is put from a height $h$ and with initial speed $v_0$, what launch angle maximizes the distance traveled? Assume that $v_0$ and $h$ are independent of the launch angle, and ignore air resistance (but include gravity, if it wasn't obvious). You can (and should) assume that $h$ is "small" and give an answer to leading non-trivial order, but you should specify what counts as "small".
\end{question}

\textbf{Pf:}

In my original attempt, after doing massive calculation, I got to the part showing the following result:
\begin{equation}
    \sin^2(\theta)\approx \frac{1}{4}\paran*{1+\sqrt{1-\frac{4gh}{v_0^2}}}
\end{equation}
\textbf{However, in my final solution, I didn't use Taylor Series approximation to get an approximated results for $\theta$, which isn't the desired form of answer}.

As a result, for $\sqrt{1-x}$, its first several derivatives is given as follow:
\begin{equation}
    \frac{d}{dx}\sqrt{1-x}=-\frac{1}{2\sqrt{1-x}},\quad\frac{d^2}{dx^2}\sqrt{1-x} = -\frac{1}{4}(1-x)^{3/2}
\end{equation}
Which, with the center about $x=0$, $\sqrt{1-x}\approx 1-\frac{1}{2}x-\frac{1}{4}x^2$ when $x$ is small; taking up to leading term, $\sqrt{1-x}\approx 1-\frac{1}{2}x$. Using this as a result, we get:
\begin{align}
    &\sin^2(\theta)\approx \frac{1}{4}\paran*{1+1-\frac{1}{2}\cdot \frac{4gh}{v_0^2}} = \frac{1}{2}-\frac{gh}{2v_0^2}\\
    &\implies \sin(\theta)\approx \frac{\sqrt{2}}{2}\sqrt{1-\frac{gh}{2v_0^2}}\approx \frac{\sqrt{2}}{2}\paran*{1-\frac{1}{2}\cdot\frac{gh}{2v_0^2}} = \frac{\sqrt{2}}{2}\paran*{1-\frac{gh}{4v_0^2}}
\end{align}


\break

\section*{4}
\begin{question}\label{q4}
    
\end{question}

\textbf{Pf:}

\break

\section*{5}
\begin{question}\label{q5}
    
\end{question}

\textbf{Pf:}

\break

\end{document}